\documentclass[reqno,12pt]{amsart}

%%%%%%%%%% Packages %%%%%%%%%%

\usepackage{amsmath,amssymb,mathtools,verbatim,bbm,kpfonts}
%\usepackage[widespace,upright]{fourier}
\usepackage[backrefs]{amsrefs}
\usepackage[protrusion=true]{microtype}
\usepackage[english]{babel}
\usepackage[margin=1in]{geometry}
\usepackage[onehalfspacing]{setspace}
\usepackage[pdfusetitle,colorlinks,pagebackref,hypertexnames=false,bookmarks=false]{hyperref}
\numberwithin{equation}{section}						% must call it before cleveref
\usepackage[nameinlink,noabbrev]{cleveref}
\expandafter\def\csname ver@etex.sty\endcsname{3000/12/31}
\let\globcount\newcount
\usepackage{autonum}									% must call after cleveref

%%%%%%%%%% Left/Right and eqref fixes %%%%%%%%%%

\let\originalleft\left
\let\originalright\right
\renewcommand{\left}{\mathopen{}\mathclose\bgroup\originalleft}
\renewcommand{\right}{\aftergroup\egroup\originalright}

\makeatletter
\renewcommand*{\eqref}[1]{\hyperref[{#1}]{\textup{\tagform@{\ref*{#1}}}}}		% eqref now links parentheses as well
\makeatother

\newcommand{\creflastconjunction}{, and\nobreakspace}		% Oxford comma in cleveref

%%%%%%%%%% Theorems/numbering %%%%%%%%%%

\newtheorem{theorem}{Theorem}[section]
\newtheorem{Mtheorem}{Main Theorem}
\newtheorem*{acknowledgment}{Acknowledgment}
\newtheorem{claim}[theorem]{Claim}
\newtheorem{condition}[theorem]{Condition}
\newtheorem{conjecture}[theorem]{Conjecture}
\newtheorem{corollary}[theorem]{Corollary}
\newtheorem{definition}[theorem]{Definition}
\newtheorem{example}[theorem]{Example}
\newtheorem{lemma}[theorem]{Lemma}
\newtheorem{proposition}[theorem]{Proposition}
\newtheorem{remark}[theorem]{Remark}
\newtheorem{hypoth}[theorem]{Hypothesis}
\crefname{theorem}{Theorem}{Theorems}						% label for Theorems
\creflabelformat{theorem}{#2{#1}#3}							% label format for 'theorem'
\crefname{Mtheorem}{Main Theorem}{Main Theorems}			% label for the Main Theorems
\creflabelformat{Mtheorem}{#2{#1}#3}						% label format for 'Mtheorem'
\crefname{lemma}{Lemma}{Lemmata}							% label for Lemmata
\creflabelformat{lemma}{#2{#1}#3}							% label format for 'lemma'
\crefname{corollary}{Corollary}{Corollaries}				% label for Corollaries
\creflabelformat{corollary}{#2{#1}#3}						% label format for 'corollary'
\crefname{proposition}{Proposition}{Propositions}			% label for Propositions
\creflabelformat{proposition}{#2{#1}#3}						% label format for 'proposition'
\crefname{ineq}{inequality}{inequalities}					% label for inequalities
\creflabelformat{ineq}{#2{\upshape(#1)}#3}					% label format for 'ineq'
\crefname{cond}{condition}{conditions}						% label for conditions
\creflabelformat{cond}{#2{\upshape(#1)}#3}					% label format for 'cond'
\crefname{hypoth}{Hypothesis}{Hypotheses}					% label for Hypotheses
\creflabelformat{hypoth}{#2{#1}#3}							% label format for 'hypoth'
\crefname{definition}{Definition}{Definitions}						% label for Definitions
\creflabelformat{def}{#2{#1}#3}								% label format for 'def'
\crefname{appsec}{Appendix}{Appendices}

%%%%%%%%%% Blackboard %%%%%%%%%%

\def\id{\mathbbm{1}}
\def\cx{\mathbb{C}}
\def\rl{\mathbb{R}}
\def\N{\mathbb{N}}
\def\P{\mathbb{P}}
\def\Z{\mathbb{Z}}

%%%%%%%%%% CalligraPhics %%%%%%%%%%

\def\cA{\mathcal{A}}
\def\cB{\mathcal{B}}
\def\cC{\mathcal{C}}
\def\cD{\mathcal{D}}
\def\cE{\mathcal{E}}
\def\cF{\mathcal{F}}
\def\cG{\mathcal{G}}
\def\cH{\mathcal{H}}
\def\cI{\mathcal{I}}
\def\cJ{\mathcal{J}}
\def\cK{\mathcal{K}}
\def\cL{\mathcal{L}}
\def\cM{\mathcal{M}}
\def\cN{\mathcal{N}}
\def\cO{\mathcal{O}}
\def\cP{\mathcal{P}}
\def\cQ{\mathcal{Q}}
\def\cR{\mathcal{R}}
\def\cS{\mathcal{S}}
\def\cT{\mathcal{T}}
\def\cU{\mathcal{U}}
\def\cV{\mathcal{V}}
\def\cW{\mathcal{W}}
\def\cZ{\mathcal{Z}}

%%%%%%%%%% Romans %%%%%%%%%%

\def\Ar{\mathrm{Area}}
\def\dist{\mathrm{dist}}
\def\Im{\mathrm{Im}}
\def\image{\mathrm{image}}
\def\Re{\mathrm{Re}}
\def\sign{\textsc{sign}}
\def\Spec{\mathrm{Spec}}
\def\supp{\mathrm{supp}}
\def\tr{\mathrm{tr}}

%%%%%%%%%% Other symbols (paper specific) %%%%%%%%%%

\def\del{\partial}
\def\delbar{\overline{\partial}}
\def\rd{\operatorname{d\!}{}}

%%%%%%%%%% Other formatting %%%%%%%%%%

\title{The Keller--Segel equation compact surfaces}
\date{\today}
\keywords{Keller--Segel equations}
%\subjclass[2010]{}

%\author{}
%\address[]{}
%\urladdr{\href{}{}}
%\email{\href{}{}}

\calclayout
\pagestyle{plain}
\clubpenalty = 10000
\widowpenalty = 10000
\setlength{\footskip}{20pt}

\hypersetup{
	unicode			= true,
	pdffitwindow	= true,
	pdftoolbar		= false,
	pdfmenubar		= false,
	pdfstartview	= {FitH},
	hypertexnames	= false,
	colorlinks		= true,
	linkcolor		= black,
	citecolor		= black,
	filecolor		= black,
	urlcolor		= blue
}

\begin{document}

\maketitle



\section{\LaTeX comments}

A few ``best practices'' that might be worth adopting (in no particular order):

\begin{itemize}
	\item Cleaner code: Try adding tabs and empty line (between paragraphs) for better readability. Nothing I ever write is good on the first try, so being able to conveniently go back and re-read the whole thing is a life-saver. :)\\
	I also like adding some amount of spaces in the the math environment, so when I have to change something later in a huge formula it's easier.

	\item Don't use $\textbackslash [ \ldots \textbackslash ]$ (or \$\$ for that matter), but the \emph{equation} environment. Better formatting, easier to add labels later, etc. You can make a macro for it if you think it's too long to type it out.

	\item Instead of arrays with an equation, use the \emph{align} environment.

	\item Less important: integrals and sums look nicer in equation mode (but not in text) if you use $\textbackslash \mathrm{limits}$.

	\item To make what I mean clearer, I've implemented the changes in your code. Let me know what you think!
\end{itemize}

\section{The Keller--Segel equation on the torus}

{\color{red} I've found a mistake in the equation that probably I've made too in the meeting. Fixed it below, please double-check!}

Consider a distribution of germs $\rho (t, x_1, x_2) = \rho (t, \underline{x})$ and food $c (t, x_1, x_2) = c (t, \underline{x})$. We impose, as a model of nature,
\begin{subequations}
\begin{align}
	\partial_t \rho	&= \partial_a^2 \rho - \partial_a \left( \rho \partial_a c \right), \label{eq:KS_1} \\
	\partial_a^2 c	&= - \rho. \label{eq:KS_2}
\end{align}
\end{subequations}
When $a$ appears as an index, summation over $a \in \{ 1, 2 \}$ is implied. We take the Fourier transform of $\rho$: For all $\underline{k} \in \mathbb{Z}^2$, let $f_{\underline{k}} (\underline{x}) = e^{2 \pi i \underline{k} \cdot \underline{x}}$. Note that $f_{\underline{k}}$ is an eigenfunction of the Laplacian; i. e. $\partial_a^2 f_{\underline{k}} = 4 \pi^2 |k^2| f_{\underline{k}}$. Also note that $\partial_a f_{\underline{k}} = 2 \pi i k_a f_{\underline{k}}$.

Now let us write
\begin{equation}
	\rho (t, \underline{x}) = \sum\limits_{\underline{k} \in \mathbb{Z}^2} R_{\underline{k}} (t) f_{\underline{k}} (\underline{x}). \label{eq:Fourier}
\end{equation}
Now any solution, $c$, to \cref{eq:KS_2} has the form
\begin{equation}
	c (t, \underline{x}) = c_0 + \sum\limits_{\substack{\underline{l} \in \mathbb{Z}^2 \\ \underline{l} \neq \underline{0}}} \frac{1}{4 \pi^2 |\underline{l}|^2} R_{\underline{l}} (t) f_{\underline{l}} (\underline{x}),
\end{equation}
where $c_0 \in \mathbb{C}$ can be chosen arbitrarily.

Using \cref{eq:KS_1,eq:Fourier} we get that
\begin{align}
	\sum\limits_{\underline{k} \in \mathbb{Z}^2} \dot{R}_{\underline{k}} f_{\underline{k}} + \sum\limits_{\underline{k} \in \mathbb{Z}^2} 4 \pi^2 |\underline{k}|^2 R_{\underline{k}} f_{\underline{k}}	&= - \partial_a \left( \sum\limits_{\underline{m} \in \mathbb{Z}^2} R_{\underline{m}} f_{\underline{m}} \sum\limits_{\substack{\underline{l} \in \mathbb{Z}^2 \\ \underline{l} \neq \underline{0}}} \frac{R_{\underline{l}}}{4 \pi^2 |\underline{l}|^2} \partial_a f_{\underline{l}} \right) \\
									&= - \sum\limits_{\substack{\underline{l}, \underline{m} \in \mathbb{Z}^2 \\ \underline{l} \neq \underline{0}}} \frac{2 \pi i l_a}{4 \pi^2 |\underline{l}|^2} R_{\underline{l}} R_{\underline{m}} \partial_a \left( f_{\underline{l}} f_{\underline{m}} \right).
\end{align}
Using that $f_{\underline{m}} f_{\underline{l}} = f_{\underline{m} + \underline{l}}$ and substituting $\underline{k} = \underline{l} + \underline{m}$ on the right-hand side, we get
\begin{align}
	\sum\limits_{\underline{k} \in \mathbb{Z}^2} \dot{R}_{\underline{k}} f_{\underline{k}} + \sum\limits_{\underline{k} \in \mathbb{Z}^2} 4 \pi^2 |\underline{k}|^2 R_{\underline{k}} f_{\underline{k}}	&= - i^2 \sum\limits_{\substack{\underline{l}, \underline{m} \in \mathbb{Z}^2 \\ \underline{l} \neq \underline{0}}} \frac{l_a (l_a + m_a)}{|\underline{l}|^2} R_{\underline{l}} R_{\underline{m}} f_{\underline{l} + \underline{m}} \\
									&= \sum\limits_{\substack{\underline{l}, \underline{m} \in \mathbb{Z}^2 \\ \underline{l} \neq \underline{0}}} \frac{\underline{l} \cdot (\underline{l} + \underline{m})}{|\underline{l}|^2} R_{\underline{l}} R_{\underline{m}} f_{\underline{l} + \underline{m}} \\
									&= \sum\limits_{\substack{\underline{l}, \underline{m} \in \mathbb{Z}^2 \\ \underline{l} \neq \underline{0}}} \frac{\underline{k} \cdot \underline{l}}{|\underline{l}|^2} R_{\underline{l}} R_{\underline{k} - \underline{l}} f_{\underline{k}}.
\end{align}
After pairing with $f_{\underline{k}}$ for any $\underline{k} \in \mathbb{Z}^2 - \{ \underline{0} \}$ and separating out $R_{\underline{k}}$ terms, we get
\begin{equation}
	\dot{R}_{\underline{k}} = \left( R_0 - 4 \pi^2 |\underline{k}|^2 \right) R_{\underline{k}} + \sum\limits_{\substack{\underline{l} \in \mathbb{Z}^2 \\ \underline{l} \neq \underline{0}, \underline{k}}} \frac{\underline{k} \cdot \underline{l}}{|\underline{l}|^2} R_{\underline{k} - \underline{l}} R_{\underline{l}}.
\end{equation}

It remains to be shown that there are such $R_{\underline{k}}$ which satisfy the equations.

\section{The general case}

Let $\Sigma$ now some compact surface, with (positive definite) Laplace operator $\Delta$ (we can discuss what that means at some point), and assume that $f_0, f_1, f_2, \ldots, f_n, \ldots$ are an orthonormal basis of eigenvectors for $L^2 (\Sigma)$, that is there are numbers $0 = \lambda_0 < \lambda_1 \leqslant \lambda_2 \leqslant \ldots \leqslant \lambda_n \leqslant \ldots$ so that for all $n, o \in N$ we have
\begin{equation}
	\Delta f_n = \lambda_n f_n, \quad \& \quad \langle f_n | f_o \rangle_{L^2 (\Sigma)} = \delta_{n, o}.
\end{equation}
For all $n, o, p \in \N$, let
\begin{equation}
	\varphi_{n, o, p} \coloneqq \int\limits_\Sigma f_n f_o f_p \rd A.
\end{equation}

\textbf{With that in mind, at some point prove the following:} If $\rho \in C^1 \left( [0, T]; L^2 (\Sigma) \right)$ solve the Keller--Segel equations on $\Sigma$ and $R_n (t) \coloneqq \langle f_n | \rho (t, \cdot) \rangle_{L^2 (\Sigma)}$, then $R_0$ is constant and
\begin{equation}
	\forall n \in \N - \{ 0 \} : \quad \dot{R}_n = \left( R_0 - \lambda_n \right) R_n + \sum\limits_{o, p \in \N - \{ 0 \}} \frac{\lambda_m - \lambda_o + \lambda_p}{\lambda_p} \varphi_{n, o, p} R_o R_p.
\end{equation}

\begin{remark}
	Note how the sign of the first term changes depending on whether $R_0 = \int_\Sigma \rho \rd A$ is small or greater than $\lambda_n$! 
\end{remark}

%===============================================================================
%\bibliography{references}
%===============================================================================

\end{document}
