\documentclass[reqno,12pt]{amsart}

%%%%%%%%%% Packages %%%%%%%%%%

\usepackage{amsmath,amssymb,mathtools,verbatim,bbm}%,kpfonts}
%\usepackage[widespace,upright]{fourier}
\usepackage[backrefs]{amsrefs}
\usepackage[protrusion=true]{microtype}
\usepackage[english]{babel}
\usepackage[margin=1in]{geometry}
\usepackage[onehalfspacing]{setspace}
\usepackage[pdfusetitle,colorlinks,pagebackref,hypertexnames=false,bookmarks=false]{hyperref}
\numberwithin{equation}{section}						% must call it before cleveref
\usepackage[nameinlink,noabbrev]{cleveref}
\expandafter\def\csname ver@etex.sty\endcsname{3000/12/31}
\let\globcount\newcount
\usepackage{autonum}									% must call after cleveref

%%%%%%%%%% Left/Right and eqref fixes %%%%%%%%%%

\let\originalleft\left
\let\originalright\right
\renewcommand{\left}{\mathopen{}\mathclose\bgroup\originalleft}
\renewcommand{\right}{\aftergroup\egroup\originalright}

\makeatletter
\renewcommand*{\eqref}[1]{\hyperref[{#1}]{\textup{\tagform@{\ref*{#1}}}}}		% eqref now links parentheses as well
\makeatother

\newcommand{\creflastconjunction}{, and\nobreakspace}		% Oxford comma in cleveref

%%%%%%%%%% Theorems/numbering %%%%%%%%%%

\newtheorem{theorem}{Theorem}[section]
\newtheorem{Mtheorem}{Main Theorem}
\newtheorem*{acknowledgment}{Acknowledgment}
\newtheorem{claim}[theorem]{Claim}
\newtheorem{condition}[theorem]{Condition}
\newtheorem{conjecture}[theorem]{Conjecture}
\newtheorem{corollary}[theorem]{Corollary}
\newtheorem{definition}[theorem]{Definition}
\newtheorem{example}[theorem]{Example}
\newtheorem{lemma}[theorem]{Lemma}
\newtheorem{proposition}[theorem]{Proposition}
\newtheorem{remark}[theorem]{Remark}
\newtheorem{hypoth}[theorem]{Hypothesis}
\crefname{theorem}{Theorem}{Theorems}						% label for Theorems
\creflabelformat{theorem}{#2{#1}#3}							% label format for 'theorem'
\crefname{Mtheorem}{Main Theorem}{Main Theorems}			% label for the Main Theorems
\creflabelformat{Mtheorem}{#2{#1}#3}						% label format for 'Mtheorem'
\crefname{lemma}{Lemma}{Lemmata}							% label for Lemmata
\creflabelformat{lemma}{#2{#1}#3}							% label format for 'lemma'
\crefname{corollary}{Corollary}{Corollaries}				% label for Corollaries
\creflabelformat{corollary}{#2{#1}#3}						% label format for 'corollary'
\crefname{proposition}{Proposition}{Propositions}			% label for Propositions
\creflabelformat{proposition}{#2{#1}#3}						% label format for 'proposition'
\crefname{ineq}{inequality}{inequalities}					% label for inequalities
\creflabelformat{ineq}{#2{\upshape(#1)}#3}					% label format for 'ineq'
\crefname{cond}{condition}{conditions}						% label for conditions
\creflabelformat{cond}{#2{\upshape(#1)}#3}					% label format for 'cond'
\crefname{hypoth}{Hypothesis}{Hypotheses}					% label for Hypotheses
\creflabelformat{hypoth}{#2{#1}#3}							% label format for 'hypoth'
\crefname{definition}{Definition}{Definitions}						% label for Definitions
\creflabelformat{def}{#2{#1}#3}								% label format for 'def'
\crefname{appsec}{Appendix}{Appendices}

%%%%%%%%%% Blackboard %%%%%%%%%%

\def\id{\mathbbm{1}}
\def\cx{\mathbb{C}}
\def\rl{\mathbb{R}}
\def\N{\mathbb{N}}
\def\P{\mathbb{P}}
\def\Z{\mathbb{Z}}

%%%%%%%%%% CalligraPhics %%%%%%%%%%

\def\cA{\mathcal{A}}
\def\cB{\mathcal{B}}
\def\cC{\mathcal{C}}
\def\cD{\mathcal{D}}
\def\cE{\mathcal{E}}
\def\cF{\mathcal{F}}
\def\cG{\mathcal{G}}
\def\cH{\mathcal{H}}
\def\cI{\mathcal{I}}
\def\cJ{\mathcal{J}}
\def\cK{\mathcal{K}}
\def\cL{\mathcal{L}}
\def\cM{\mathcal{M}}
\def\cN{\mathcal{N}}
\def\cO{\mathcal{O}}
\def\cP{\mathcal{P}}
\def\cQ{\mathcal{Q}}
\def\cR{\mathcal{R}}
\def\cS{\mathcal{S}}
\def\cT{\mathcal{T}}
\def\cU{\mathcal{U}}
\def\cV{\mathcal{V}}
\def\cW{\mathcal{W}}
\def\cZ{\mathcal{Z}}

%%%%%%%%%% Romans %%%%%%%%%%

\def\Ar{\mathrm{Area}}
\def\dist{\mathrm{dist}}
\def\Im{\mathrm{Im}}
\def\image{\mathrm{image}}
\def\Re{\mathrm{Re}}
\def\sign{\textsc{sign}}
\def\Spec{\mathrm{Spec}}
\def\supp{\mathrm{supp}}
\def\tr{\mathrm{tr}}

%%%%%%%%%% Other symbols (paper specific) %%%%%%%%%%

\def\del{\partial}
\def\delbar{\overline{\partial}}
\def\rd{\operatorname{d\!}{}}

%%%%%%%%%% Other formatting %%%%%%%%%%

\title{The Keller--Segel equation on the torus}
\date{\today}
\keywords{Keller--Segel equations}
%\subjclass[2010]{}

%\author{}
%\address[]{}
%\urladdr{\href{}{}}
%\email{\href{}{}}

\calclayout
\pagestyle{plain}
\clubpenalty = 10000
\widowpenalty = 10000
\setlength{\footskip}{20pt}

\hypersetup{
	unicode			= true,
	pdffitwindow	= true,
	pdftoolbar		= false,
	pdfmenubar		= false,
	pdfstartview	= {FitH},
	hypertexnames	= false,
	colorlinks		= true,
	linkcolor		= black,
	citecolor		= black,
	filecolor		= black,
	urlcolor		= blue
}

\begin{document}

\maketitle

%We can use this file for our notes. You can compile with xelatex or pdflatex.

%(You can also use the shell script I wrote for compilation, compile.sh, but it's probably an overkill at this stage. Also I did not make a documentation for it.)

Consider a distribution of germs $\rho(t,x_1,x_2)=\rho(t,\underline x)$ and food $c(t,x_1,x_2)=c(t,\underline x)$.  We impose, as a model of nature,
\[\left\{\begin{array}l\partial_t\rho=\partial_a^2\rho-\partial_a(\rho\partial_ac),\\\partial_a^2c=-\rho.\end{array}\right.\]
When $a$ appears as an index, summation over $a\in\{x_1,x_2\}$ is implied.  We take the fourier transform of $\rho$: with $f_{k_1,k_2}(x_1,x_2)=f_{\underline k}(\underline x)=e^{2\pi i\underline k\cdot\underline x}$ eigenfunctions,%!ad!why don't we use Einstein summation for fourier transforms?!ad!
\[\rho(t,\underline x)=\sum_{\underline k\in\mathbb Z^2}R_{\underline k}(t)f_{\underline k}(\underline x).\]
Now we may explicitly write $c$ to satisfy the shorter equation:
\[c(t,\underline x)=\sum_{\substack{\underline k\in\mathbb Z^2\\\underline k\neq\underline0}}\frac1{4\pi^2|\underline k|^2}R_{\underline k}(t)f_{\underline k}(\underline x).\]%!ad!perhaps some clarification about $c$ not being unique!ad!
Let's write the remaining equation more explicitly:
\[\sum_{\underline k\in\mathbb Z^2}\dot{R_{\underline k}}f_{\underline k}=-\sum_{\underline k\in\mathbb Z^2}4\pi^2|\underline k|^2R_{\underline k}f_{\underline k}-\partial_a\left(\sum_{\underline k\in\mathbb Z^2}\sum_{\substack{\underline l\in\mathbb Z^2\\\underline l\neq\underline0}}\frac{R_{\underline k}f_{\underline k}}{4\pi^2|\underline l|^2}2\pi ik_aR_{\underline l}f_{\underline l}\right).\]
Next we perform the outer derivative and rearrange:
\[\sum_{\underline k\in\mathbb Z^2}\dot{R_{\underline k}}f_{\underline k}+\sum_{\underline k\in\mathbb Z^2}4\pi^2|\underline k|^2R_{\underline k}f_{\underline k}=-i^2\sum_{\underline k\in\mathbb Z^2}\sum_{\substack{\underline l\in\mathbb Z^2\\\underline l\neq\underline0}}\frac{R_{\underline k}R_{\underline l}f_{\underline k+\underline l}}{|\underline l|^2}k_a(k_a+l_a).\]
We substitute $k\mapsto k-l$ in the right sum so that
\[\sum_{\underline k\in\mathbb Z^2}\left(\dot{R_{\underline k}}+4\pi^2|\underline k|^2R_{\underline k}\right)f_{\underline k}=\sum_{\underline k\in\mathbb Z^2}\sum_{\substack{\underline l\in\mathbb Z^2\\\underline l\neq\underline0}}\frac{R_{\underline k-\underline l}R_{\underline l}}{|\underline l|^2}\underline k\cdot(\underline k-\underline l)f_{\underline k}.\]
We use the fact, as we have implicitly already used to assert the form of $\rho$, that $L^2(\mathbb T)$ is a vector space with basis $\{f_k\}$ to assert that the coefficients are equal
\[\forall\underline k\quad\dot{R_{\underline k}}+4\pi^2|\underline k|^2R_{\underline k}=\sum_{\substack{\underline l\in\mathbb Z^2\\\underline l\neq\underline0}}\frac{R_{\underline k-\underline l}R_{\underline l}}{|\underline l|^2}\underline k\cdot(\underline k-\underline l).\]
It remains to be shown that there are such $R_{\underline k}$ which satisfy the equations.

%===============================================================================
%\bibliography{references}
%===============================================================================

\end{document}
