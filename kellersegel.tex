\documentclass[reqno,12pt]{amsart}

%%%%%%%%%% Packages %%%%%%%%%%

\usepackage{amsmath,amssymb,mathtools,verbatim,bbm,kpfonts}
%\usepackage[widespace,upright]{fourier}
\usepackage[backrefs]{amsrefs}
\usepackage[protrusion=true,babel=true]{microtype}
\usepackage[english]{babel}
\usepackage[margin=1in]{geometry}
\usepackage[onehalfspacing]{setspace}
\usepackage[pdfusetitle,colorlinks,pagebackref,hypertexnames=false,bookmarks=false]{hyperref}
\numberwithin{equation}{section}						% must call it before cleveref
\usepackage[nameinlink,noabbrev]{cleveref}
\expandafter\def\csname ver@etex.sty\endcsname{3000/12/31}
\let\globcount\newcount
\usepackage{autonum}									% must call after cleveref

%%%%%%%%%% Left/Right and eqref fixes %%%%%%%%%%

\let\originalleft\left
\let\originalright\right
\renewcommand{\left}{\mathopen{}\mathclose\bgroup\originalleft}
\renewcommand{\right}{\aftergroup\egroup\originalright}

\makeatletter
\renewcommand*{\eqref}[1]{\hyperref[{#1}]{\textup{\tagform@{\ref*{#1}}}}}		% eqref now links parentheses as well
\makeatother

\newcommand{\creflastconjunction}{, and\nobreakspace}		% Oxford comma in cleveref

%%%%%%%%%% Theorems/numbering %%%%%%%%%%

\newtheorem{theorem}{Theorem}[section]
\newtheorem{Mtheorem}{Main Theorem}
\newtheorem*{acknowledgment}{Acknowledgment}
\newtheorem{claim}[theorem]{Claim}
\newtheorem{condition}[theorem]{Condition}
\newtheorem{conjecture}[theorem]{Conjecture}
\newtheorem{corollary}[theorem]{Corollary}
\newtheorem{definition}[theorem]{Definition}
\newtheorem{example}[theorem]{Example}
\newtheorem{lemma}[theorem]{Lemma}
\newtheorem{proposition}[theorem]{Proposition}
\newtheorem{remark}[theorem]{Remark}
\newtheorem{hypoth}[theorem]{Hypothesis}
\crefname{theorem}{Theorem}{Theorems}						% label for Theorems
\creflabelformat{theorem}{#2{#1}#3}							% label format for 'theorem'
\crefname{Mtheorem}{Main Theorem}{Main Theorems}			% label for the Main Theorems
\creflabelformat{Mtheorem}{#2{#1}#3}						% label format for 'Mtheorem'
\crefname{lemma}{Lemma}{Lemmata}							% label for Lemmata
\creflabelformat{lemma}{#2{#1}#3}							% label format for 'lemma'
\crefname{corollary}{Corollary}{Corollaries}				% label for Corollaries
\creflabelformat{corollary}{#2{#1}#3}						% label format for 'corollary'
\crefname{proposition}{Proposition}{Propositions}			% label for Propositions
\creflabelformat{proposition}{#2{#1}#3}						% label format for 'proposition'
\crefname{ineq}{inequality}{inequalities}					% label for inequalities
\creflabelformat{ineq}{#2{\upshape(#1)}#3}					% label format for 'ineq'
\crefname{cond}{condition}{conditions}						% label for conditions
\creflabelformat{cond}{#2{\upshape(#1)}#3}					% label format for 'cond'
\crefname{hypoth}{Hypothesis}{Hypotheses}					% label for Hypotheses
\creflabelformat{hypoth}{#2{#1}#3}							% label format for 'hypoth'
\crefname{definition}{Definition}{Definitions}						% label for Definitions
\creflabelformat{def}{#2{#1}#3}								% label format for 'def'
\crefname{appsec}{Appendix}{Appendices}

%%%%%%%%%% Blackboard %%%%%%%%%%

\def\id{\mathbbm{1}}
\def\cx{\mathbb{C}}
\def\rl{\mathbb{R}}
\def\N{\mathbb{N}}
\def\P{\mathbb{P}}
\def\Z{\mathbb{Z}}

%%%%%%%%%% CalligraPhics %%%%%%%%%%

\def\cA{\mathcal{A}}
\def\cB{\mathcal{B}}
\def\cC{\mathcal{C}}
\def\cD{\mathcal{D}}
\def\cE{\mathcal{E}}
\def\cF{\mathcal{F}}
\def\cG{\mathcal{G}}
\def\cH{\mathcal{H}}
\def\cI{\mathcal{I}}
\def\cJ{\mathcal{J}}
\def\cK{\mathcal{K}}
\def\cL{\mathcal{L}}
\def\cM{\mathcal{M}}
\def\cN{\mathcal{N}}
\def\cO{\mathcal{O}}
\def\cP{\mathcal{P}}
\def\cQ{\mathcal{Q}}
\def\cR{\mathcal{R}}
\def\cS{\mathcal{S}}
\def\cT{\mathcal{T}}
\def\cU{\mathcal{U}}
\def\cV{\mathcal{V}}
\def\cW{\mathcal{W}}
\def\cZ{\mathcal{Z}}

%%%%%%%%%% Romans %%%%%%%%%%

\def\Ar{\mathrm{Area}}
\def\dist{\mathrm{dist}}
\def\Im{\mathrm{Im}}
\def\image{\mathrm{image}}
\def\Re{\mathrm{Re}}
\def\sign{\textsc{sign}}
\def\Spec{\mathrm{Spec}}
\def\supp{\mathrm{supp}}
\def\tr{\mathrm{tr}}

%%%%%%%%%% Other symbols (paper specific) %%%%%%%%%%

\def\del{\partial}
\def\delbar{\overline{\partial}}
\def\rd{\operatorname{d\!}{}}

%%%%%%%%%% Other formatting %%%%%%%%%%

\title{The Keller--Segel equation compact surfaces}
\date{\today}

\calclayout
\pagestyle{plain}
\clubpenalty = 10000
\widowpenalty = 10000
\setlength{\footskip}{20pt}

\hypersetup{
	unicode			= true,
	pdffitwindow	= true,
	pdftoolbar		= false,
	pdfmenubar		= false,
	pdfstartview	= {FitH},
	hypertexnames	= false,
	colorlinks		= true,
	linkcolor		= black,
	citecolor		= black,
	filecolor		= black,
	urlcolor		= blue
}

\newcommand{\ul}{\underline l}
\newcommand{\uk}{\underline k}
\newcommand{\uj}{\underline j}

\begin{document}

\maketitle



\section{The Keller--Segel equation on the torus}

Consider a distribution of germs $\rho (t, x_1, x_2) = \rho (t, \underline{x})$ and food $c (t, x_1, x_2) = c (t, \underline{x})$. We impose, as a model of nature,
\begin{subequations}
\begin{align}
	\partial_t \rho	&= \partial_a^2 \rho - \partial_a \left( \rho \partial_a c \right), \label{eq:KS_1} \\
	\partial_a^2 c	&= - \rho. \label{eq:KS_2}
\end{align}
\end{subequations}
When $a$ appears as an index, summation over $a \in \{ 1, 2 \}$ is implied. We take the Fourier transform of $\rho$: For all $\underline{k} \in \Z^2$, let $f_{\underline{k}} (\underline{x}) = e^{2 \pi i \underline{k} \cdot \underline{x}}$. Note that $f_{\underline{k}}$ is an eigenfunction of the Laplacian; i.e. $\partial_a^2 f_{\underline{k}} = -4 \pi^2 |k|^2 f_{\underline{k}}$. Also note that $\partial_a f_{\underline{k}} = 2 \pi i k_a f_{\underline{k}}$.%toremove: i.e. vs i. e.?

Let us write
\begin{equation}
	\rho (t, \underline{x}) = \sum\limits_{\underline{k} \in \Z^2} R_{\underline{k}} (t) f_{\underline{k}} (\underline{x}). \label{eq:Fourier}
\end{equation}
Now any solution, $c$, to \cref{eq:KS_2} has the form
\begin{equation}
	c (t, \underline{x}) = c_0 + \sum\limits_{\substack{\underline{l} \in \Z^2 \\ \underline{l} \neq \underline{0}}} \frac{1}{4 \pi^2 |\underline{l}|^2} R_{\underline{l}} (t) f_{\underline{l}} (\underline{x}),
\end{equation}
where $c_0 \in \cx$ can be chosen arbitrarily.

Using \cref{eq:KS_1,eq:Fourier} we get that
\begin{align}
	\sum\limits_{\underline{k} \in \Z^2} \dot{R}_{\underline{k}} f_{\underline{k}} + \sum\limits_{\underline{k} \in \Z^2} 4 \pi^2 |\underline{k}|^2 R_{\underline{k}} f_{\underline{k}}	&= - \partial_a \left( \sum\limits_{\underline{m} \in \Z^2} R_{\underline{m}} f_{\underline{m}} \sum\limits_{\substack{\underline{l} \in \Z^2 \\ \underline{l} \neq \underline{0}}} \frac{R_{\underline{l}}}{4 \pi^2 |\underline{l}|^2} \partial_a f_{\underline{l}} \right) \\
									&= - \sum\limits_{\substack{\underline{l}, \underline{m} \in \Z^2 \\ \underline{l} \neq \underline{0}}} \frac{2 \pi i l_a}{4 \pi^2 |\underline{l}|^2} R_{\underline{l}} R_{\underline{m}} \partial_a \left( f_{\underline{l}} f_{\underline{m}} \right).
\end{align}
Using that $f_{\underline{m}} f_{\underline{l}} = f_{\underline{m} + \underline{l}}$ and substituting $\underline{k} = \underline{l} + \underline{m}$ on the right-hand side, we get
\begin{align}
	\sum\limits_{\underline{k} \in \Z^2} \dot{R}_{\underline{k}} f_{\underline{k}} + \sum\limits_{\underline{k} \in \Z^2} 4 \pi^2 |\underline{k}|^2 R_{\underline{k}} f_{\underline{k}}	&= - i^2 \sum\limits_{\substack{\underline{l}, \underline{m} \in \Z^2 \\ \underline{l} \neq \underline{0}}} \frac{l_a (l_a + m_a)}{|\underline{l}|^2} R_{\underline{l}} R_{\underline{m}} f_{\underline{l} + \underline{m}} \\
									&= \sum\limits_{\substack{\underline{l}, \underline{m} \in \Z^2 \\ \underline{l} \neq \underline{0}}} \frac{\underline{l} \cdot (\underline{l} + \underline{m})}{|\underline{l}|^2} R_{\underline{l}} R_{\underline{m}} f_{\underline{l} + \underline{m}} \\
									&= \sum\limits_{\substack{\underline{l}, \underline{k} \in \Z^2 \\ \underline{l} \neq \underline{0}}} \frac{\underline{l} \cdot \underline{k}}{|\underline{l}|^2} R_{\underline{l}} R_{\underline{k} - \underline{l}} f_{\underline{k}}.
\end{align}
After pairing with $f_{\underline{k}}$ for any $\underline{k} \in \Z^2 - \{ \underline{0} \}$ and separating out $R_{\underline{k}}$ terms, we get
\begin{equation}
	\dot{R}_{\underline{k}} = \left( R_0 - 4 \pi^2 |\underline{k}|^2 \right) R_{\underline{k}} + \sum\limits_{\substack{\underline{l} \in \Z^2 \\ \underline{l} \neq \underline{0}, \underline{k}}} \frac{\underline{l} \cdot \underline{k}}{|\underline{l}|^2} R_{\underline{l}} R_{\underline{k} - \underline{l}}.
\end{equation}

Let us consider solutions with analytic $R_{\uk}$.  Taking $R_{\uk}(t)=\sum_{i=0}^\infty R_{\uk,i}t^i$, we have

\begin{equation}
	\sum_{i=0}^\infty (i+1)R_{\uk,i+1}t^i = \left( R_0 - 4 \pi^2 |\uk|^2 \right) \sum_{i=0}^\infty R_{\uk,i}t^i + \sum\limits_{\substack{\ul \in \Z^2 \\ \ul \neq \underline{0}, \uk}} \frac{\ul \cdot \uk}{|\ul|^2} \sum_{i=0}^\infty\sum_{j=0}^\infty R_{\ul,i} R_{\uk - \ul,j}t^{i+j}.
\end{equation}
Equating coefficients of powers of $t$, we obtain

\begin{equation}
	R_{\uk,i+1} = \frac1{i+1}\left(\left( R_0 - 4 \pi^2 |\uk|^2 \right) R_{\uk,i} + \sum\limits_{\substack{\ul \in \Z^2 \\ \ul \neq \underline{0}, \uk}} \frac{\ul \cdot \uk}{|\ul|^2} \sum_{j=0}^i R_{\ul,j} R_{\uk - \ul,i-j}\right)
\end{equation}
which is an explicit recursive formula for the $R_k$'s given intitial coefficients.  This formula may be used for numerical computation; we might have polynomial initial conditions and produce simple series solutions for the $R_k$'s.  Along this line of inquiry, we prove the following theorem about the number of terms in successive truncations: it may at most double.

\theorem{}
Suppose for fixed $i$ and $D$, $R_{\uk,I}=0$ for $I\leq i$ and $|\uk|_\infty>D$.  Then $R_{\uk,i+1}=0$ for $|\uk|_\infty>2D$.
\proof
We have
\begin{align}
	(i+1)R_{\uk,i+1} 
		&= \left( R_0 - 4 \pi^2 |\uk|^2 \right) R_{\uk,i} + \sum\limits_{\substack{\ul \in \Z^2 \\ \ul \neq \underline{0}, \uk}} \frac{\ul \cdot \uk}{|\ul|^2} \sum_{j=0}^i R_{\ul,j} R_{\uk - \ul,i-j}\\
		&= \left( R_0 - 4 \pi^2 |\uk|^2 \right) R_{\uk,i} + \sum\limits_{\substack{\ul \in \Z^2 \\ |\ul|_\infty\leq D \\ \ul \neq \underline{0}, \uk}} \frac{\ul \cdot \uk}{|\ul|^2} \sum_{j=0}^i R_{\ul,j} R_{\uk - \ul,i-j}.
\end{align}
Take $|\uk|_\infty>2D$.  Since $|\ul|_\infty\leq D\leq|\uk|_\infty$, the reverse triangle inequality yields
\begin{equation}
	|\uk-\ul|_\infty\geq|\uk|_\infty-|\ul|_\infty>2D-|\ul|_\infty=D+(D-|\ul|_\infty)\geq D+0=D
\end{equation}
where the rightmost inequality follows from $|l|_\infty\leq D$.  Then $R_{\uk-\ul,i-j}=0$ for all $i-j$.  Of course $R_{\uk,i}=0$ by assumption, so $(i+1)R_{\uk,i+1}=0$.

\section{The general case}

Let $\Sigma$ now some compact surface, with (positive definite) Laplace operator $\Delta$ (we can discuss what that means at some point), and assume that $f_0, f_1, f_2, \ldots, f_n, \ldots$ are an orthonormal basis of eigenvectors for $L^2 (\Sigma)$, that is there are numbers $0 = \lambda_0 < \lambda_1 \leqslant \lambda_2 \leqslant \ldots \leqslant \lambda_n \leqslant \ldots$ so that for all $n, o \in N$ we have
\begin{equation}
	\Delta f_n = \lambda_n f_n, \quad \& \quad \langle f_n | f_o \rangle_{L^2 (\Sigma)} = \delta_{n, o}.
\end{equation}
For all $n, o, p \in \N$, let
\begin{equation}
	\varphi_{n, o, p} \coloneqq \int\limits_\Sigma f_n f_o f_p \rd A.
\end{equation}

\textbf{With that in mind, at some point prove the following:} If $\rho \in C^1 \left( [0, T]; L^2 (\Sigma) \right)$ solve the Keller--Segel equations on $\Sigma$ and $R_n (t) \coloneqq \langle f_n | \rho (t, \cdot) \rangle_{L^2 (\Sigma)}$, then $R_0$ is constant and
\begin{equation}
	\forall n \in \N - \{ 0 \} : \quad \dot{R}_n = \left( R_0 - \lambda_n \right) R_n + \sum\limits_{o, p \in \N - \{ 0 \}} \frac{\lambda_m - \lambda_o + \lambda_p}{\lambda_p} \varphi_{n, o, p} R_o R_p.
\end{equation}

\begin{remark}
	Note how the sign of the first term changes depending on whether $R_0 = \int_\Sigma \rho \rd A$ is small or greater than $\lambda_n$! 
\end{remark}

\section{Banach Business}
Consider the map $\cK:\{R_{\underline k}\}_{\underline k\in\Z^2}\to C$ ({\color{red} $C$ is something, perhaps $\{R_{\underline k}\}_{\underline k\in\Z^2}$}) defined by
\begin{equation}
	\cK(R_{\underline k})=R_{\underline k}(0)+%I forget what this, written as little r_{\underline k}, was supposed to be
		\int\limits_0^t (R_0-4\pi^2|\underline k|^2)R_{\underline k}(\tau)+\sum\limits_{\substack{\underline{l} \in \Z^2 \\ \underline{l} \neq \underline{0}}}\frac{\underline k\cdot\underline l}{|\underline l|^2}R_{\underline k-\underline l}(\tau)R_{\underline l}(\tau) \rd\tau
\end{equation}
Collections of fourier coefficients with $\cK(R)=R$ satisfy \cref{eq:KS_1}.  We seek, therefore, conditions (on $(R_{\underline k})_{\underline k\in\Z^2}$ and $t$) which yield fixed points of $\cK$.  It would be sufficient to bound
\begin{equation}
	\lVert\cK(R)-\cK(S)\rVert\leq\theta\lVert R-S\rVert
\end{equation}
where $\lVert\cdot\rVert$ is perhaps
\begin{equation}
	\sup_{t\in[0,t]} \sum_{\underline k\in\Z^2} |R_k(t)|^2.
\end{equation}
and $0\leq\theta<1$ is some constant.  We might begin to consider $\lVert\cK(R)-\cK(S)\rVert=$
\begin{equation}
	\left\lVert
		\int\limits_0^t (R_0-4\pi^2|\underline k|^2) \left(R_{\underline k}(\tau)-S_{\underline k}(\tau)\right) + 
			\sum\limits_{\substack{\underline{l} \in \Z^2 \\ \underline{l} \neq \underline{0}}}
			\frac{\underline k\cdot\underline l}{|\underline l|^2} \left(R_{\underline k-\underline l}(\tau)R_{\underline l}(\tau) - S_{\underline k-\underline l}(\tau)S_{\underline l}(\tau)\right)
		\rd\tau
	\right\rVert.
\end{equation}
It may be fruitful to bound the summands of $\lVert\cdot\rVert$ corresponding to values of $\underline k$, sharply enough that the infinite sum converges.

\section{Bounding Blunders}
Circuitously, we will show $\sum_{\underline k\in[1,2,\ldots]^2}\frac1{|\underline k|^2}=\infty$.  See that
\begin{equation}
	\sum_{n=1}^\infty\frac1{n^2+m^2}=\frac{\coth(\pi m)\pi m-1}{2m^2}.
\end{equation}
Now $\coth(x)>\frac1{\pi x}+\frac{\pi x}3-\frac{\pi^3x^3}{45}$ so that
\begin{align}
	\sum_{n,m=1}^\infty\frac1{n^2+m^2}=\sum_{m=1}^\infty\frac{\coth(\pi m)\pi m-1}{2m^2}&>\sum_{m=1}^\infty\frac{\left(\frac1{\pi^2m}+\frac{\pi^2m}3-\frac{\pi^6m^3}{45}\right)\pi m-1}{2m^2}\\
	&=\sum_{m=1}^\infty\frac{1+\frac{\pi^4m^2}3-\frac{\pi^8m^4}{45}-\pi}{2\pi m^2}=\infty.
\end{align}
Details may be filled in later.  In the mean time, I will think of other sums.

We have
\begin{equation}
	S(p)=\sum_{\substack{n_0=1\\n_1=1}}^\infty\frac1{n_0^{2p}+n_1^{2p}}
		=\frac{-p+n_1\pi\sum_{i=1}^p(-1)^{\frac{2i-1}{2p}}\cot\left((-1)^{\frac{2i-1}{2p}}n_1\pi\right)}{2pn_1^{2p}}
\end{equation}
for $p>0$ an integer.  Consistent with $\zeta$, for odd powers of $n_0$ the sum is more difficult to evaluate.  We expand the cotangents into infinite sums and combine:
\begin{align}
		S(p)&=\frac{-p+n_1\pi\sum_{i=1}^p(-1)^{\frac{2i-1}{2p}}\sum_{j=0}^\infty\frac{(-1)^j2^{2j}B_{2j}}{(2j)!}\left((-1)^{\frac{2i-1}{2p}}n_1\pi\right)^{2j-1}}{2pn_1^{2p}}\\
		&=\frac{-p+n_1\pi\sum_{j=0}^\infty(n_1\pi)^{2j-1}\frac{(-1)^j2^{2j}B_{2j}}{(2j)!}\sum_{i=1}^p(-1)^{\frac{2i-1}{2p}}(-1)^{\frac{(2i-1)(2j-1)}{2p}}}{2pn_1^{2p}}\\
		&=\frac{-p+n_1\pi\sum_{j=0}^\infty(n_1\pi)^{2j-1}\frac{(-1)^j2^{2j}B_{2j}}{(2j)!}\sum_{i=1}^p(-1)^{\frac{(2i-1)j}p}}{2pn_1^{2p}}.
\end{align}
This particular sum of roots of unity is a number theoretic one.  It evaluates to $(-1)^{\frac jp}p$ if $p$ divides $j$, and $0$ if $p$ does not divide $j$.  That is,
\begin{equation}
	\sum_{i=1}^p(-1)^{\frac{(2i-1)j}p}=\left\{\begin{array}{cl}(-1)^{\frac jp}p&\text{if }p|j\\0&\text{if }p\nmid j.\end{array}\right.
\end{equation}
Then
\begin{align}
	S(p)&=\frac{-p+n_1\pi\sum_{k=0}^\infty(n_1\pi)^{2pk-1}\frac{(-1)^{pk}2^{2pk}B_{2pk}}{(2pk)!}(-1)^kp}{2pn_1^{2p}}\\
	&=\frac{-1+\sum_{k=0}^\infty(n_1\pi)^{2pk}\frac{(-1)^{(p+1)k}2^{2pk}B_{2pk}}{(2pk)!}}{2n_1^{2p}}
\end{align}

\section{Symmetric in $R$ summands}

We have
\begin{equation}
	\dot{R}_{\uk} = \left(R_0-4\pi^2|\uk|^2\right)R_{\uk} + \sum\limits_{\substack{\ul\in\Z^2\\\ul\neq\underline0,\uk}} \frac{\ul\cdot\uk}{|\ul|^2}R_{\ul}R_{\uk-\ul}.
\end{equation}
Substitute $\uj=\ul-\frac\uk2$.  Then

\begin{equation}
	\dot{R}_{\uk} = \left(R_0-4\pi^2|\uk|^2\right)R_{\uk} + \sum\limits_{\substack{\uj\in\Z^2+\frac{\text{mod}(\uk,2)}2\\\uj\neq\pm\frac\uk2}} \frac{(\uj+\frac\uk2)\cdot\uk}{|\uj+\frac\uk2|^2}R_{\frac\uk2+\uj}R_{\frac\uk2-\uj}
\end{equation}
where $\text{mod}(n)$ is zero for even $n$ and 1 for odd $n$ (that is, $\uj$ and $\frac\uk2$ lie on common integer grids with offsets of $0$ or $\frac12$) (and it is evaluated elementwise).  Let us make each summand have unique $R$ products: we only sum over $j$ with $0\leq\arctan\uj<\pi$:

\begin{equation}
	\dot{R}_{\uk} = \left(R_0-4\pi^2|\uk|^2\right)R_{\uk} + \sum\limits_{\substack{0\leq\arctan\uj<\pi\\\uj\in\Z^2+\frac{\text{mod}(\uk,2)}2\\\uj\neq\pm\frac\uk2}} \left(\frac{(\uj+\frac\uk2)\cdot\uk}{|\uj+\frac\uk2|^2}+\frac{(-\uj+\frac\uk2)\cdot\uk}{|-\uj+\frac\uk2|^2}\right)R_{\frac\uk2+\uj}R_{\frac\uk2-\uj}.
\end{equation}
We write the coefficients more explicitly:

\begin{equation}
	\dot{R}_{\uk} = \left(R_0-4\pi^2|\uk|^2\right)R_{\uk} + \sum\limits_{\substack{0\leq\arctan\uj<\pi\\\uj\in\Z^2+\frac{\text{mod}(\uk,2)}2\\\uj\neq\pm\frac\uk2}} \frac{|\uk|^2\left(|\uj|^2+\frac{|\uk|^2}4\right)-2(\uj\cdot\uk)^2}{\left(|\uj|^2-\uj\cdot\uk+\frac{|\uk|^2}4\right)\left(|\uj|^2+\uj\cdot\uk+\frac{|\uk|^2}4\right)} R_{\frac\uk2+\uj}R_{\frac\uk2-\uj}.
\end{equation}

%===============================================================================
%\bibliography{references}
%===============================================================================

\end{document}
