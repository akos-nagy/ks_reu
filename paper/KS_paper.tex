\documentclass[12pt,reqno]{amsart}

%%%%%%%%%% Packages %%%%%%%%%%

\usepackage{amsmath,amssymb,mathtools,verbatim,enumitem,appendix}%,kpfonts}
\usepackage[widespace,upright]{fourier}
\usepackage[backrefs]{amsrefs}
\usepackage[protrusion=true,babel=true]{microtype}
\usepackage[english]{babel}
\usepackage[margin=1in]{geometry}
\usepackage[onehalfspacing]{setspace}
\usepackage[pdfusetitle,colorlinks,pagebackref,hypertexnames=false,bookmarks=false]{hyperref}
\numberwithin{equation}{section}							% must call it before cleveref
\usepackage[nameinlink,noabbrev]{cleveref}
\expandafter\def\csname ver@etex.sty\endcsname{3000/12/31}	% this fixes a random, irrelevant warning that clever throws
\let\globcount\newcount
\usepackage{autonum}										% must call after cleveref

%%%%%%%%%% align break fix %%%%%%%%%%

\allowdisplaybreaks

%%%%%%%%%% Left/Right fix %%%%%%%%%%

\let\originalleft\left
\let\originalright\right
\renewcommand{\left}{\mathopen{}\mathclose\bgroup\originalleft}
\renewcommand{\right}{\aftergroup\egroup\originalright}

%%%%%%%%%% eqref fix %%%%%%%%%%

\makeatletter
\renewcommand*{\eqref}[1]{\hyperref[{#1}]{\textup{\tagform@{\ref*{#1}}}}}
\makeatother

%%%%%%%%%% oxford comma fix %%%%%%%%%%

\newcommand{\creflastconjunction}{, and\nobreakspace}

%%%%%%%%%% coloneqq fix %%%%%%%%%%

\newcommand{\eqdef}{\mathrel{\vcenter{\baselineskip0.5ex\lineskiplimit0pt\hbox{.}\hbox{.}}}=}

%%%%%%%%%% Theorems/numbering %%%%%%%%%%

\newtheorem{theorem}{Theorem}[section]
\newtheorem{Mtheorem}{Main Theorem}
\newtheorem*{acknowledgment}{Acknowledgment}
\newtheorem{claim}[theorem]{Claim}
\newtheorem{condition}[theorem]{Condition}
\newtheorem{conjecture}[theorem]{Conjecture}
\newtheorem{corollary}[theorem]{Corollary}
\newtheorem{definition}[theorem]{Definition}
\newtheorem{example}[theorem]{Example}
\newtheorem{lemma}[theorem]{Lemma}
\newtheorem{proposition}[theorem]{Proposition}
\newtheorem{remark}[theorem]{Remark}
\newtheorem{hypoth}[theorem]{Hypothesis}
\crefname{theorem}{Theorem}{Theorems}						% label for Theorems
\creflabelformat{theorem}{#2{#1}#3}							% label format for 'theorem'
\crefname{Mtheorem}{Main Theorem}{Main Theorems}			% label for the Main Theorems
\creflabelformat{Mtheorem}{#2{#1}#3}						% label format for 'Mtheorem'
\crefname{lemma}{Lemma}{Lemmata}							% label for Lemmata
\creflabelformat{lemma}{#2{#1}#3}							% label format for 'lemma'
\crefname{corollary}{Corollary}{Corollaries}				% label for Corollaries
\creflabelformat{corollary}{#2{#1}#3}						% label format for 'corollary'
\crefname{proposition}{Proposition}{Propositions}			% label for Propositions
\creflabelformat{proposition}{#2{#1}#3}						% label format for 'proposition'
\crefname{ineq}{inequality}{inequalities}					% label for inequalities
\creflabelformat{ineq}{#2{\upshape(#1)}#3}					% label format for 'ineq'
\crefname{cond}{condition}{conditions}						% label for conditions
\creflabelformat{cond}{#2{\upshape(#1)}#3}					% label format for 'cond'
\crefname{hypoth}{Hypothesis}{Hypotheses}					% label for Hypotheses
\creflabelformat{hypoth}{#2{#1}#3}							% label format for 'hypoth'
\crefname{definition}{Definition}{Definitions}						% label for Definitions
\creflabelformat{def}{#2{#1}#3}								% label format for 'def'
\crefname{appsec}{Appendix}{Appendices}

%%%%%%%%%% Blackboard %%%%%%%%%%

\def\id{\mathbbm{1}}
\def\cx{\mathbb{C}}
\def\rl{\mathbb{R}}
\def\N{\mathbb{N}}
\def\P{\mathbb{P}}
\def\Z{\mathbb{Z}}

%%%%%%%%%% CalligraPhics %%%%%%%%%%

\def\cA{\mathcal{A}}
\def\cB{\mathcal{B}}
\def\cC{\mathcal{C}}
\def\cD{\mathcal{D}}
\def\cE{\mathcal{E}}
\def\cF{\mathcal{F}}
\def\cG{\mathcal{G}}
\def\cH{\mathcal{H}}
\def\cI{\mathcal{I}}
\def\cJ{\mathcal{J}}
\def\cK{\mathcal{K}}
\def\cL{\mathcal{L}}
\def\cM{\mathcal{M}}
\def\cN{\mathcal{N}}
\def\cO{\mathcal{O}}
\def\cP{\mathcal{P}}
\def\cQ{\mathcal{Q}}
\def\cR{\mathcal{R}}
\def\cS{\mathcal{S}}
\def\cT{\mathcal{T}}
\def\cU{\mathcal{U}}
\def\cV{\mathcal{V}}
\def\cW{\mathcal{W}}
\def\cZ{\mathcal{Z}}

%%%%%%%%%% Romans %%%%%%%%%%

\def\Ar{\mathrm{Area}}
\def\dist{\mathrm{dist}}
\def\Im{\mathrm{Im}}
\def\image{\mathrm{image}}
\def\Re{\mathrm{Re}}
\def\sign{\textsc{sign}}
\def\Spec{\mathrm{Spec}}
\def\supp{\mathrm{supp}}
\def\tr{\mathrm{tr}}

%%%%%%%%%% Other symbols (paper specific) %%%%%%%%%%

\def\del{\partial}
\def\delbar{\overline{\partial}}
\def\rd{\operatorname{d\!}{}}

%%%%%%%%%% Other formatting %%%%%%%%%%

\title{The Keller--Segel equation compact surfaces}
\date{\today}
\keywords{Keller--Segel equations}
\keywords{chemotaxis, Keller--Segel equations}
\subjclass[2020]{35J15, 35Q92, 92C17}

\author{Adam Mendenhall}
\address[UCSB]{University of California, Santa Barbara}
\email{\href{amendenhall@ucsb.edu}{amendenhall@ucsb.edu}}

\author{\'Akos Nagy}
\address[UCSB]{University of California, Santa Barbara}
\email{\href{contact@akosnagy.com}{contact@akosnagy.com}}
\urladdr{\href{https://akosnagy.com}{akosnagy.com}}

\calclayout
\pagestyle{plain}
\clubpenalty = 10000
\widowpenalty = 10000
\setlength{\footskip}{20pt}

\hypersetup{
	unicode			= true,
	pdffitwindow	= true,
	pdftoolbar		= false,
	pdfmenubar		= false,
	pdfstartview	= {FitH},
	hypertexnames	= false,
	colorlinks		= true,
	linkcolor		= black,
	citecolor		= black,
	filecolor		= black,
	urlcolor		= blue
}

\newcommand{\ul}{\underline l}
\newcommand{\uk}{\underline k}
\newcommand{\uj}{\underline j}

\begin{document}

\begin{abstract}
	We study the parabolic--elliptic Keller--Segel equations on compact surfaces.

	[rest of to be completed...]
\end{abstract}

\maketitle

\section{Introduction}

{\color{red} This is the part that we will write last.}

\medskip

\subsection*{Organization of the paper}

To be completed...

\medskip

\begin{acknowledgment}
	To be completed...
\end{acknowledgment}

\bigskip

\section{Reformulation of the equation}

Let $\Sigma$ be a smooth, closed surface with a Riemannian metric $g$ and area form $\omega$. Let $G$ be the Green operator of the Laplacian, $\Delta$, on $L^2 \left( \Sigma, g \right)$ and let us fix $\varrho_0 \in L^2 \left( \Sigma, g \right)$. Note that $L^2 \hookrightarrow L^1$ on domains of finite measure. We say that a positive function, $\varrho \in C^1 \left( (0, T); L^2 \left( \Sigma, g \right) \right)$ is said to satisfy the \emph{parabolic--elliptic Keller--Segel equations} on $\left( \Sigma, g \right)$ with initial value $\varrho_0$ if it is a solution to the following system:
\begin{subequations}
\begin{align}
	\del_t \varrho								&= - \Delta \varrho + \rd^* \left( \varrho \rd G \left( \varrho \right) \right), \label{eq:KS_eq} \\
	\lim\limits_{t \rightarrow 0^+} \varrho_t	&= \varrho_0. \label{eq:KS_IV}
\end{align}
\end{subequations}
where for all $t \in (0, T)$
\begin{equation}
	 \varrho_t \eqdef \varrho|_{\{ t \} \times \Sigma},
\end{equation}
regarded as a function in $L^2 \left( \Sigma, g \right)$.

Now let
\begin{equation}
	m (t) \eqdef \int\limits_\Sigma \varrho_t \: \omega.
\end{equation}
Then $m$ is constant, because
\begin{equation}
	\dot{m} (t) = \int\limits_\Sigma \left( - \Delta \varrho + \rd^* \left( \varrho \rd G \left( \varrho \right) \right) \right) \: \omega = \int\limits_\Sigma \rd^* \left( - \rd \varrho + \varrho \rd G \left( \varrho \right) \right) \: \omega = 0.
\end{equation}
Thus we drop the $t$-dependence of $m$ from its notation.

\smallskip

For the rest of the paper, let $A_\Sigma \eqdef \Ar \left( \Sigma, g \right)$. The following lemma recasts \cref{eq:KS_eq,eq:KS_IV} in a simpler form.

\begin{lemma}
	Let $\chi_0 \coloneqq \varrho_0 - \tfrac{m}{A_\Sigma}$ and $\chi \coloneqq \varrho - \tfrac{m}{A_\Sigma}$. Then \cref{eq:KS_eq,eq:KS_IV} is equivalent to
	\begin{subequations}
	\begin{align}
		\del_t \chi						&= \left( \tfrac{m}{A_\Sigma} - \Delta \right) \chi + \rd^* \left( \chi \rd G \left( \chi \right) \right), \label{eq:new_KS_eq} \\
		\chi|_{\{ 0 \} \times \Sigma}	&= \chi_0. \label{eq:new_KS_IV}
	\end{align}
	\end{subequations}
\end{lemma}

\begin{proof}
	To be completed...
\end{proof}

\medskip

\section{The iteration}

Let now $\left( \Psi_a \right)_{a \in \N}$ be a orthonormal eigenbasis of $\Delta$ and
\begin{equation}
	\Delta \Psi_a = \lambda_a \Psi_a.
\end{equation}
Let us order this basis so that
\begin{equation}
	0 = \lambda_0 < \lambda_1 \leqslant \lambda_2 \leqslant \ldots \lambda_a \leqslant \lambda_{a + 1} \leqslant \ldots
\end{equation}
In particular, $\Psi_0 = \tfrac{1}{\sqrt{A_\Sigma}}$.

Assume that $\chi$ is a solution to \cref{eq:new_KS_eq,eq:new_KS_IV} and write
\begin{equation}
	R_a (t) \eqdef \left\langle \Psi_a \middle| \chi|_{\{ t \} \times \Sigma} \right\rangle_{L^2 \left( \Sigma, g \right)}.
\end{equation}
Then for each $a \in \N$, we have that $R_a \in C^1 \left( (0, T); \rl \right)$. Note that $R_0 \equiv \tfrac{m}{A_\Sigma}$. Finally let
\begin{equation}
	\forall a, b, c \in \N : \quad \varphi_{abc} \coloneqq \int\limits_\Sigma \Psi_a \Psi_b \Psi_c \: \omega.
\end{equation}

\begin{theorem}
	The function $\varrho$ is a solution to \cref{eq:KS_eq,eq:KS_IV} exactly when the functions $\left( R_a \right)_{a \in \N}$ satisfy
	\begin{subequations}
	\begin{align}
		\forall t \in (0, T) : \quad \left( R_a (t) \right)_{a \in \N} &\in l^2 \left( \N \right), \\
		\varrho_0	&= \sum\limits_{a \in \N} \left( \lim\limits_{t \rightarrow 0^+} R_a (t) \right) \Psi_a, \\
		\forall a \in \N - \{ 0 \} : \quad \dot{R}_a	&= \left( \tfrac{m}{A_\Sigma} - \lambda_a \right) R_a + \sum\limits_{b, c \in N} \frac{\lambda_a - \lambda_b + \lambda_c}{\lambda_c} \varphi_{abc} R_b R_c.
	\end{align}
	\end{subequations}
\end{theorem}


	%=========================
	%\bibliography{references}
	%=========================

\end{document}
