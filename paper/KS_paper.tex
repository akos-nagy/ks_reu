\documentclass[12pt,reqno]{amsart}

%%%%%%%%%% Packages %%%%%%%%%%

\usepackage{amsmath,amssymb,mathtools,verbatim,enumitem,appendix,kpfonts}
%\usepackage[widespace,upright]{fourier}
\usepackage[backrefs]{amsrefs}
\usepackage[protrusion=true,babel=true]{microtype}
\usepackage[english]{babel}
\usepackage[margin=1in]{geometry}
\usepackage[onehalfspacing]{setspace}
\usepackage[pdfusetitle,colorlinks,pagebackref,hypertexnames=false,bookmarks=false]{hyperref}
\numberwithin{equation}{section}							% must call it before cleveref
\usepackage[nameinlink,noabbrev]{cleveref}
\expandafter\def\csname ver@etex.sty\endcsname{3000/12/31}	% this fixes a random, irrelevant warning that clever throws
\let\globcount\newcount
\usepackage{autonum}										% must call after cleveref
\usepackage[T1]{fontenc}

%%%%%%%%%% align break fix %%%%%%%%%%

\allowdisplaybreaks

%%%%%%%%%% Left/Right fix %%%%%%%%%%

\let\originalleft\left
\let\originalright\right
\renewcommand{\left}{\mathopen{}\mathclose\bgroup\originalleft}
\renewcommand{\right}{\aftergroup\egroup\originalright}

%%%%%%%%%% eqref fix %%%%%%%%%%

\makeatletter
\renewcommand*{\eqref}[1]{\hyperref[{#1}]{\textup{\tagform@{\ref*{#1}}}}}
\makeatother

%%%%%%%%%% oxford comma fix %%%%%%%%%%

\newcommand{\creflastconjunction}{, and\nobreakspace}

%%%%%%%%%% coloneqq fix %%%%%%%%%%

\newcommand{\eqdef}{\mathrel{\vcenter{\baselineskip0.5ex\lineskiplimit0pt\hbox{.}\hbox{.}}}=}

%%%%%%%%%% Theorems/numbering %%%%%%%%%%

\newtheorem{theorem}{Theorem}[section]
\newtheorem{Mtheorem}{Main Theorem}
\newtheorem*{acknowledgment}{Acknowledgment}
\newtheorem{claim}[theorem]{Claim}
\newtheorem{condition}[theorem]{Condition}
\newtheorem{conjecture}[theorem]{Conjecture}
\newtheorem{corollary}[theorem]{Corollary}
\newtheorem{definition}[theorem]{Definition}
\newtheorem{example}[theorem]{Example}
\newtheorem{lemma}[theorem]{Lemma}
\newtheorem{proposition}[theorem]{Proposition}
\newtheorem{remark}[theorem]{Remark}
\newtheorem{hypoth}[theorem]{Hypothesis}
\crefname{theorem}{Theorem}{Theorems}						% label for Theorems
\creflabelformat{theorem}{#2{#1}#3}							% label format for 'theorem'
\crefname{Mtheorem}{Main Theorem}{Main Theorems}			% label for the Main Theorems
\creflabelformat{Mtheorem}{#2{#1}#3}						% label format for 'Mtheorem'
\crefname{lemma}{Lemma}{Lemmata}							% label for Lemmata
\creflabelformat{lemma}{#2{#1}#3}							% label format for 'lemma'
\crefname{corollary}{Corollary}{Corollaries}				% label for Corollaries
\creflabelformat{corollary}{#2{#1}#3}						% label format for 'corollary'
\crefname{proposition}{Proposition}{Propositions}			% label for Propositions
\creflabelformat{proposition}{#2{#1}#3}						% label format for 'proposition'
\crefname{ineq}{inequality}{inequalities}					% label for inequalities
\creflabelformat{ineq}{#2{\upshape(#1)}#3}					% label format for 'ineq'
\crefname{cond}{condition}{conditions}						% label for conditions
\creflabelformat{cond}{#2{\upshape(#1)}#3}					% label format for 'cond'
\crefname{hypoth}{Hypothesis}{Hypotheses}					% label for Hypotheses
\creflabelformat{hypoth}{#2{#1}#3}							% label format for 'hypoth'
\crefname{definition}{Definition}{Definitions}						% label for Definitions
\creflabelformat{def}{#2{#1}#3}								% label format for 'def'
\crefname{appsec}{Appendix}{Appendices}

%%%%%%%%%% Blackboard %%%%%%%%%%

\def\id{\mathbbm{1}}
\def\cx{\mathbb{C}}
\def\rl{\mathbb{R}}
\def\N{\mathbb{N}}
\def\P{\mathbb{P}}
\def\Z{\mathbb{Z}}

%%%%%%%%%% CalligraPhics %%%%%%%%%%

\def\cA{\mathcal{A}}
\def\cB{\mathcal{B}}
\def\cC{\mathcal{C}}
\def\cD{\mathcal{D}}
\def\cE{\mathcal{E}}
\def\cF{\mathcal{F}}
\def\cG{\mathcal{G}}
\def\cH{\mathcal{H}}
\def\cI{\mathcal{I}}
\def\cJ{\mathcal{J}}
\def\cK{\mathcal{K}}
\def\cL{\mathcal{L}}
\def\cM{\mathcal{M}}
\def\cN{\mathcal{N}}
\def\cO{\mathcal{O}}
\def\cP{\mathcal{P}}
\def\cQ{\mathcal{Q}}
\def\cR{\mathcal{R}}
\def\cS{\mathcal{S}}
\def\cT{\mathcal{T}}
\def\cU{\mathcal{U}}
\def\cV{\mathcal{V}}
\def\cW{\mathcal{W}}
\def\cZ{\mathcal{Z}}

%%%%%%%%%% Romans %%%%%%%%%%

\def\Ar{\mathrm{Area}}
\def\dist{\mathrm{dist}}
\def\Im{\mathrm{Im}}
\def\image{\mathrm{image}}
\def\Re{\mathrm{Re}}
\def\sign{\textsc{sign}}
\def\Spec{\mathrm{Spec}}
\def\supp{\mathrm{supp}}
\def\tr{\mathrm{tr}}

%%%%%%%%%% Other symbols (paper specific) %%%%%%%%%%

\def\del{\partial}
\def\delbar{\overline{\partial}}
\def\rd{\operatorname{d\!}{}}
\def\dA{\: \rd A}

%%%%%%%%%% Other formatting %%%%%%%%%%

\title{The Keller--Segel equation compact surfaces}
\date{\today}
\keywords{Keller--Segel equations}
\keywords{chemotaxis, Keller--Segel equations}
\subjclass[2020]{35J15, 35Q92, 92C17}

\author{Adam Mendenhall}
\address[UCSB]{University of California, Santa Barbara}
\email{\href{amendenhall@ucsb.edu}{amendenhall@ucsb.edu}}

\author{\'Akos Nagy}
\address[UCSB]{University of California, Santa Barbara}
\email{\href{contact@akosnagy.com}{contact@akosnagy.com}}
\urladdr{\href{https://akosnagy.com}{akosnagy.com}}

\calclayout
\pagestyle{plain}
\clubpenalty = 10000
\widowpenalty = 10000
\setlength{\footskip}{20pt}

\hypersetup{
	unicode			= true,
	pdffitwindow	= true,
	pdftoolbar		= false,
	pdfmenubar		= false,
	pdfstartview	= {FitH},
	hypertexnames	= false,
	colorlinks		= true,
	linkcolor		= black,
	citecolor		= black,
	filecolor		= black,
	urlcolor		= blue
}

\newcommand{\ul}{\underline l}
\newcommand{\uk}{\underline k}
\newcommand{\uj}{\underline j}

\begin{document}

\begin{abstract}
	We study the (parabolic-elliptic) Keller--Segel equations on compact surfaces.

	[rest of to be completed...]
\end{abstract}

\maketitle

\section{Introduction}

{\color{red} This is the part that we will write last.}

\medskip

\subsection*{Organization of the paper}

To be completed...

\medskip

\begin{acknowledgment}
	To be completed...
\end{acknowledgment}

\bigskip

\section{Reformulation of the equation}

Let $\Sigma$ be a smooth, closed surface with a Riemannian metric $g$ and area form $\omega$. Let $G$ be the Green operator of the Laplacian, $\Delta$, on $L^2 \left( \Sigma, g \right)$ and let us fix $\varrho_0 \in L^2 \left( \Sigma, g \right)$. Note that $L^2 \hookrightarrow L^1$ on domains of finite measure. We say that a positive function, $\varrho \in C^1 \left( (0, T); L^2 \left( \Sigma, g \right) \right)$ is said to satisfy the \emph{parabolic--elliptic Keller--Segel equations} on $\left( \Sigma, g \right)$ with initial value $\varrho_0$ if it is a solution to the following system:
\begin{subequations}
\begin{align}
	\del_t \varrho								&= - \Delta \varrho + \rd^* \left( \varrho \rd G \left( \varrho \right) \right), \label{eq:KS_eq} \\
	\lim\limits_{t \rightarrow 0^+} \varrho_t	&= \varrho_0. \label{eq:KS_IV}
\end{align}
\end{subequations}
where for all $t \in (0, T)$
\begin{equation}
	 \varrho_t \eqdef \varrho|_{\{ t \} \times \Sigma},
\end{equation}
regarded as a function in $L^2 \left( \Sigma, g \right)$.

Now let
\begin{equation}
	m (t) \eqdef \int\limits_\Sigma \varrho_t \dA.
\end{equation}
Then $m$ is constant, because
\begin{equation}
	\dot{m} (t) = \int\limits_\Sigma \left( - \Delta \varrho + \rd^* \left( \varrho \rd G \left( \varrho \right) \right) \right) \dA = \int\limits_\Sigma \rd^* \left( - \rd \varrho + \varrho \rd G \left( \varrho \right) \right) \dA = 0.
\end{equation}
Thus we drop the $t$-dependence of $m$ from its notation.

\smallskip

For the rest of the paper, let $A_\Sigma \eqdef \Ar \left( \Sigma, g \right)$. The following lemma recasts \cref{eq:KS_eq,eq:KS_IV} in a simpler form.

\begin{lemma}
	Let $\chi_0 \coloneqq \varrho_0 - \tfrac{m}{A_\Sigma}$ and $\chi \coloneqq \varrho - \tfrac{m}{A_\Sigma}$. Then \cref{eq:KS_eq,eq:KS_IV} is equivalent to
	\begin{subequations}
	\begin{align}
		\del_t \chi						&= \left( \tfrac{m}{A_\Sigma} - \Delta \right) \chi + \rd^* \left( \chi \rd G \left( \chi \right) \right), \label{eq:new_KS_eq} \\
		\chi|_{\{ 0 \} \times \Sigma}	&= \chi_0. \label{eq:new_KS_IV}
	\end{align}
	\end{subequations}
\end{lemma}

\begin{proof}
	The equivalency of \cref{eq:KS_IV} and \cref{eq:new_KS_IV} is obvious.

	Note that $G$ annihilates constants and since $\chi$ is orthogonal to constants, we have that $\Delta \left( G \left( \chi \right) \right) = \chi$. Since $\chi = \varrho - \tfrac{m}{A_\Sigma}$, we get, using \cref{eq:KS_eq}, that
	\begin{align}
		\del_t \chi	&= \del_t \left( \varrho - \tfrac{m}{A_\Sigma} \right) \\
					&= \del_t \varrho - 0 \\
					&= - \Delta \varrho + \rd^* \left( \varrho \rd G \left( \varrho \right) \right) \\
					&= - \Delta \left( \tfrac{m}{A_\Sigma} + \chi \right) + \rd^* \left( \left( \tfrac{m}{A_\Sigma} + \chi \right) \rd G \left( \tfrac{m}{A_\Sigma} + \chi \right) \right) \\
					&= - \Delta \chi + \rd^* \left( \left( \tfrac{m}{A_\Sigma} + \chi \right) \rd G \left( \chi \right) \right) \\
					&= - \Delta \chi + \tfrac{m}{A_\Sigma} \rd^* \rd G \left( \chi \right) + \rd^* \left( \chi \rd G \left( \chi \right) \right) \\
					&= \left( \tfrac{m}{A_\Sigma} - \Delta \right) \chi + \rd^* \left( \chi \rd G \left( \chi \right) \right),
	\end{align}
	which completes the proof.
\end{proof}

\smallskip

\begin{remark}
	Let $\lambda_1$ be the smallest nonzero eigenvalue of $\Delta$ and note that quantity $M_\Sigma = \lambda_1 A_\Sigma$ only depends on the geometry of $\left( \Sigma, g \right)$. When $m < M_\Sigma$, then the linear term in \cref{eq:new_KS_eq} is strictly negative definite.
\end{remark}

\medskip

\section{The generalized Fourier transform}

Let now $\left( \Psi_a \in L^2 \left( \Sigma, g \right) \right)_{a \in \N}$ be an orthonormal eigenbasis of $\Delta$ and
\begin{equation}
	\Delta \Psi_a = \lambda_a \Psi_a.
\end{equation}
Let us order this basis so that
\begin{equation}
	0 = \lambda_0 < \lambda_1 \leqslant \lambda_2 \leqslant \ldots \lambda_a \leqslant \lambda_{a + 1} \leqslant \ldots
\end{equation}
In particular, $\Psi_0 = \tfrac{1}{\sqrt{A_\Sigma}}$.

Assume that $\chi$ is a solution to \cref{eq:new_KS_eq,eq:new_KS_IV} and write
\begin{equation}
	R_a (t) \eqdef \left\langle \Psi_a \middle| \chi|_{\{ t \} \times \Sigma} \right\rangle_{L^2 \left( \Sigma, g \right)}.
\end{equation}
Then for each $a \in \N$, we have that $R_a \in C^1 \left( (0, T); \rl \right)$. Note that $R_0 \equiv \tfrac{m}{A_\Sigma}$. Finally let
\begin{equation}
	\forall a, b, c \in \N : \quad \varphi_{abc} \coloneqq \int\limits_\Sigma \Psi_a \Psi_b \Psi_c \dA.
\end{equation}
Note that $\varphi_{abc}$ is a completely symmetric 3-tensor.

\smallskip

\begin{theorem}
\label{theorem:ODE}
	The function $\varrho$ is a solution to \cref{eq:KS_eq} exactly when
	\begin{subequations}
	\begin{align}
		\forall t \in (0, T) &: \quad \left( R_a (t) \right)_{a \in \N} \in l^2 \left( \N \right), \label[cond]{cond:R_l2} \\
		\forall a \in \N_+ &: \quad \dot{R}_a = \left( \tfrac{m}{A_\Sigma} - \lambda_a \right) R_a + \sum\limits_{b, c \in \N} \frac{\lambda_a - \lambda_b + \lambda_c}{2 \lambda_c} \varphi_{abc} R_b R_c. \label{eq:R_a_eq}
	\end{align}
	\end{subequations}
	Furthermore, if \cref{eq:KS_IV} is also satisfied, then for all $a \in \N$, $\lim_{t \rightarrow 0^+} R_a (t)$ exists and
	\begin{equation}
		\varrho_0 = \sum\limits_{a \in \N} \left( \lim\limits_{t \rightarrow 0^+} R_a (t) \right) \Psi_a. \label{eq:R_a_IV}
	\end{equation}
\end{theorem}

\begin{proof}
	Since $\left( \Psi_a \right)_{a \in \N}$ is an orthonormal basis of $L^2 \left( \Sigma, g \right)$ and for all $t \in (0, T)$, $\varrho_t$ is in $L^2 \left( \Sigma, g \right)$, we get \cref{cond:R_l2}.

	Fix $a \in \N_+$. Then
	\begin{equation}
		G \left( \Psi_a \right) = \lambda_a^{- 1} \Psi_a.
	\end{equation}
	Using the above equation, the self-adjointness of $\Delta$, and \cref{eq:new_KS_eq}, we get
	\begin{align}
		\dot{R}_a	&= \left\langle \Psi_a \middle| \partial_t \chi \right\rangle_{L^2 \left( \Sigma, g \right)} \\
					&= \left\langle \Psi_a \middle| \left( \tfrac{m}{A_\Sigma} - \Delta \right) \chi + \rd^* \left( \chi \rd G \left( \chi \right) \right) \right\rangle_{L^2 \left( \Sigma, g \right)} \\
					&= \left\langle \left( \tfrac{m}{A_\Sigma} - \Delta \right) \Psi_a \middle| \chi \right\rangle_{L^2 \left( \Sigma, g \right)} + \left\langle \Psi_a \middle| \rd^* \left( \chi \rd G \left( \chi \right) \right) \right\rangle_{L^2 \left( \Sigma, g \right)} \\
					&= \left( \tfrac{m}{A_\Sigma} - \lambda_a \right) \left\langle \Psi_a \middle| \chi \right\rangle_{L^2 \left( \Sigma, g \right)} + \sum\limits_{b, c \in \N_+} \left\langle \Psi_a \middle| \rd^* \left( \Psi_b \rd G \left( \Psi_c \right) \right) \right\rangle_{L^2 \left( \Sigma, g \right)} R_b R_c \\
					&= \left( \tfrac{m}{A_\Sigma} - \lambda_a \right) R_a + \sum\limits_{b, c \in \N_+} \left\langle \rd \Psi_a \middle| \Psi_b \rd \Psi_c \right\rangle_{L^2 \left( \Sigma, g \right)} \lambda_c^{- 1} R_b R_c. \label{eq:coeffs}
	\end{align}
	Note that
	\begin{equation}
		\left\langle \rd \Psi_a \middle| \Psi_b \rd \Psi_c \right\rangle_{L^2 \left( \Sigma, g \right)} = \int\limits_\Sigma \Psi_b g \left( \rd \Psi_a, \rd \Psi_c \right) \dA, \label{eq:triple_coeff}
	\end{equation}
	and
	\begin{equation}
		\Delta \left( \Psi_a \Psi_c \right) = \left( \Delta \Psi_a \right) \Psi_c + \Psi_a \left( \Delta \Psi_c \right) - 2 g \left( \rd \Psi_a, \rd \Psi_c \right) = \left( \lambda_a + \lambda_c \right) \Psi_a \Psi_c - 2 g \left( \rd \Psi_a, \rd \Psi_c \right).
	\end{equation}
	Thus
	\begin{equation}
		g \left( \rd \Psi_a, \rd \Psi_c \right) = \frac{\lambda_a + \lambda_c}{2} \Psi_a \Psi_c - \frac{1}{2} \Delta \left( \Psi_a \Psi_c \right).
	\end{equation}
	Plugging the above equation into \cref{eq:triple_coeff} we get
	\begin{align}
		\left\langle \rd \Psi_a \middle| \Psi_b \rd \Psi_c \right\rangle_{L^2 \left( \Sigma, g \right)}	&= \int\limits_\Sigma \Psi_b g \left( \rd \Psi_a, \rd \Psi_c \right) \dA \\
			&= \int\limits_\Sigma \Psi_b \left( \frac{\lambda_a + \lambda_c}{2} \Psi_a \Psi_c - \frac{1}{2} \Delta \left( \Psi_a \Psi_c \right) \right) \dA \\
			&= \frac{\lambda_a + \lambda_c}{2} \varphi_{abc} - \frac{1}{2} \int\limits_\Sigma \left( \Delta \Psi_b \right) \Psi_a \Psi_c \dA \\
			&= \frac{\lambda_a + \lambda_c}{2} \varphi_{abc} - \frac{\lambda_b}{2} \int\limits_\Sigma \Psi_b \Psi_a \Psi_c \dA \\
			&=  \frac{\lambda_a - \lambda_b + \lambda_c}{2} \varphi_{abc}.
	\end{align}
	Inserting this into \cref{eq:coeffs} yields \cref{eq:R_a_eq}.

	The equivalency of \cref{eq:KS_IV} and \cref{eq:R_a_IV} is straightforward.
\end{proof}

\smallskip

\begin{remark}
	The moral of \Cref{theorem:ODE} is that the Keller--Segel equations, which is a (hard) elliptic-parabolic system of partial differential equations, can be transformed (on closed surfaces) into a infinite system of ordinary differential equations, which is potentially easier to handle.

	In the rest of the paper we show that this system can be further simplified under certain extra hypotheses.
\end{remark}

\smallskip

\subsection{Analytic solutions}

In order to further simplify \cref{eq:R_a_eq}, we search for analytic solutions, that is
\begin{equation}
	\forall a \in \N : \forall t \in (0, T): \quad R_a (t) = \sum\limits_{n \in \N} R_{a, n} t^n,
\end{equation}
and the right-hand side is assumed to be absolute convergent in $l^2 \left( \N \right)$.

The next lemma rewrites \cref{eq:R_a_eq} in terms of the coefficients $\left( R_{a, n} \right)_{(a, n) \in \N \times \N}$.

\begin{lemma}
	Under the above assumption, the function $\varrho$ is a $t$-analytic solution to \cref{eq:KS_eq} with mass $m$ exactly when $R_{0, 0} = \tfrac{m}{A_\Sigma}$, for all $n \in \N_+$, $R_{0, n} = 0$, and
	\begin{subequations}
	\begin{align}
		\forall a, n \in \N	&: \quad R_{a, n + 1} = \frac{1}{n + 1} \left( \left( \tfrac{m}{A_\Sigma} - \lambda_a \right) R_{a, n} + \sum\limits_{b, c \in \N} \sum\limits_{m = 0}^n \frac{\lambda_a - \lambda_b + \lambda_c}{2 \lambda_c} \varphi_{abc} R_{b, m} R_{c, n - m} \right), \label{eq:general_iteration} \\
		\forall n \in \N	&: \quad \cR_n \coloneqq \left( R_{a, n} \right)_{a \in \N} \in l^2 \left( \N \right), \label{eq:regularity} \\
		& \qquad \limsup\limits_{n \rightarrow \infty} \| \cR_n \|_{l^2 \left( \N \right)}^{\frac{1}{n}} = \limsup\limits_{n \rightarrow \infty} \left( \sum\limits_{a \in \N} R_{a, n}^2 \right)^{\frac{1}{n}} < \frac{1}{T}. \label{eq:convergence_radius}
	\end{align}
	\end{subequations}
\end{lemma}

\begin{proof}
	To be completed...
\end{proof}

\bigskip

\section{Round spheres}

To be completed...

\bigskip

\section{Flat tori}

To be completed...

\bigskip

\section{Solutions on $\rl^2 / \Z^2$}

*If we figure this out.*

	%=========================
	%\bibliography{references}
	%=========================

\end{document}
