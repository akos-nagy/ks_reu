\documentclass[12pt,reqno]{amsart}

%%%%%%%%%% Packages %%%%%%%%%%

\usepackage{amsmath,amssymb,mathtools,verbatim,enumitem,appendix,bbm}%,kpfonts}
\usepackage[widespace,upright]{fourier}
\usepackage[backrefs]{amsrefs}
\usepackage[protrusion=true,babel=true]{microtype}
\usepackage[english]{babel}
\usepackage[margin=1in]{geometry}
\usepackage[onehalfspacing]{setspace}
\usepackage[pdfusetitle,colorlinks,pagebackref,hypertexnames=false,bookmarks=false]{hyperref}
\numberwithin{equation}{section}							% must call it before cleveref
\usepackage[nameinlink,noabbrev]{cleveref}
\expandafter\def\csname ver@etex.sty\endcsname{3000/12/31}	% this fixes a random, irrelevant warning that clever throws
\let\globcount\newcount
\usepackage{autonum}										% must call after cleveref
%\usepackage[T1]{fontenc}

%%%%%%%%%% align break fix %%%%%%%%%%

\allowdisplaybreaks

%%%%%%%%%% Left/Right fix %%%%%%%%%%

\let\originalleft\left
\let\originalright\right
\renewcommand{\left}{\mathopen{}\mathclose\bgroup\originalleft}
\renewcommand{\right}{\aftergroup\egroup\originalright}

%%%%%%%%%% eqref fix %%%%%%%%%%

\makeatletter
\renewcommand*{\eqref}[1]{\hyperref[{#1}]{\textup{\tagform@{\ref*{#1}}}}}
\makeatother

%%%%%%%%%% oxford comma fix %%%%%%%%%%

\newcommand{\creflastconjunction}{, and\nobreakspace}

%%%%%%%%%% eqdef fix %%%%%%%%%%

\newcommand{\eqdef}{\mathrel{\vcenter{\hbox{.}\hbox{.}}}=}
\newcommand{\defeq}{=\mathrel{\vcenter{\baselineskip0.5ex\lineskiplimit0pt\hbox{.}\hbox{.}}}}

%%%%%%%%%% Theorems/numbering %%%%%%%%%%

\newtheorem{theorem}{Theorem}[section]
\newtheorem{Mtheorem}{Main Theorem}
\newtheorem*{acknowledgment}{Acknowledgment}
\newtheorem{claim}[theorem]{Claim}
\newtheorem{condition}[theorem]{Condition}
\newtheorem{conjecture}[theorem]{Conjecture}
\newtheorem{corollary}[theorem]{Corollary}
\newtheorem{definition}[theorem]{Definition}
\newtheorem{example}[theorem]{Example}
\newtheorem{lemma}[theorem]{Lemma}
\newtheorem{proposition}[theorem]{Proposition}
\newtheorem{remark}[theorem]{Remark}
\newtheorem{hypoth}[theorem]{Hypothesis}
\crefname{theorem}{Theorem}{Theorems}					% label for Theorems
\creflabelformat{theorem}{#2{#1}#3}						% label format for 'theorem'
\crefname{Mtheorem}{Main Theorem}{Main Theorems}		% label for the Main Theorems
\creflabelformat{Mtheorem}{#2{#1}#3}					% label format for 'Mtheorem'
\crefname{lemma}{Lemma}{Lemmata}						% label for Lemmata
\creflabelformat{lemma}{#2{#1}#3}						% label format for 'lemma'
\crefname{corollary}{Corollary}{Corollaries}			% label for Corollaries
\creflabelformat{corollary}{#2{#1}#3}					% label format for 'corollary'
\crefname{theorem}{Proposition}{Propositions}			% label for Propositions
\creflabelformat{theorem}{#2{#1}#3}						% label format for 'theorem'
\crefname{ineq}{inequality}{inequalities}				% label for inequalities
\creflabelformat{ineq}{#2{\upshape(#1)}#3}				% label format for 'ineq'
\crefname{cond}{condition}{conditions}					% label for conditions
\creflabelformat{cond}{#2{\upshape(#1)}#3}				% label format for 'cond'
\crefname{hypoth}{Hypothesis}{Hypotheses}				% label for Hypotheses
\creflabelformat{hypoth}{#2{#1}#3}						% label format for 'hypoth'
\crefname{definition}{Definition}{Definitions}			% label for Definitions
\creflabelformat{def}{#2{#1}#3}							% label format for 'def'
\crefname{appsec}{Appendix}{Appendices}

%%%%%%%%%% Blackboard %%%%%%%%%%

\def\id{\mathbbm{1}}
\def\cx{\mathbb{C}}
\def\rl{\mathbb{R}}
\def\N{\mathbb{N}}
\def\P{\mathbb{P}}
\def\Z{\mathbb{Z}}

%%%%%%%%%% CalligraPhics %%%%%%%%%%

\def\cA{\mathcal{A}}
\def\cB{\mathcal{B}}
\def\cC{\mathcal{C}}
\def\cD{\mathcal{D}}
\def\cE{\mathcal{E}}
\def\cF{\mathcal{F}}
\def\cG{\mathcal{G}}
\def\cH{\mathcal{H}}
\def\cI{\mathcal{I}}
\def\cJ{\mathcal{J}}
\def\cK{\mathcal{K}}
\def\cL{\mathcal{L}}
\def\cM{\mathcal{M}}
\def\cN{\mathcal{N}}
\def\cO{\mathcal{O}}
\def\cP{\mathcal{P}}
\def\cQ{\mathcal{Q}}
\def\cR{\mathcal{R}}
\def\cS{\mathcal{S}}
\def\cT{\mathcal{T}}
\def\cU{\mathcal{U}}
\def\cV{\mathcal{V}}
\def\cW{\mathcal{W}}
\def\cZ{\mathcal{Z}}

%%%%%%%%%% Romans %%%%%%%%%%

\def\Ar{\mathrm{Area}}
\def\dist{\mathrm{dist}}
\def\Im{\mathrm{Im}}
\def\image{\mathrm{image}}
\def\Re{\mathrm{Re}}
\def\sign{\textsc{sign}}
\def\Spec{\mathrm{Spec}}
\def\supp{\mathrm{supp}}
\def\tr{\mathrm{tr}}

%%%%%%%%%% Other symbols (paper specific) %%%%%%%%%%

\def\del{\partial}
\def\delbar{\overline{\partial}}
\def\rd{\operatorname{d\!}{}}
\def\dA{\: \rd A}
\def\loc{\mathrm{loc}}
\def\TBC{\underline{\textbf{To be completed.}}}

%%%%%%%%%% Other formatting %%%%%%%%%%

\title{The Keller--Segel equation closed surfaces}
\date{\today}
\keywords{Keller--Segel equations}
\keywords{chemotaxis, Keller--Segel equations}
\subjclass[2020]{35J15, 35Q92, 92C17}

\author{Adam Mendenhall}
\address[Adam Mendenhall]{University of California, Santa Barbara}
\email{\href{amendenhall@ucsb.edu}{amendenhall@ucsb.edu}}

\author{\'Akos Nagy}
\address[\'Akos Nagy]{University of California, Santa Barbara}
\email{\href{contact@akosnagy.com}{contact@akosnagy.com}}
\urladdr{\href{https://akosnagy.com}{akosnagy.com}}

\calclayout
\pagestyle{plain}
\clubpenalty = 10000
\widowpenalty = 10000
\setlength{\footskip}{20pt}

\hypersetup{
	unicode			= true,
	pdffitwindow	= true,
	pdftoolbar		= false,
	pdfmenubar		= false,
	pdfstartview	= {FitH},
	hypertexnames	= false,
	colorlinks		= true,
	linkcolor		= black,
	citecolor		= black,
	filecolor		= black,
	urlcolor		= blue
}

\newcommand{\ul}{\underline l}
\newcommand{\uk}{\underline k}
\newcommand{\uj}{\underline j}

\begin{document}

\begin{abstract}
	We study the (parabolic-elliptic) Keller--Segel equations on closed surfaces.

	\TBC
\end{abstract}

\maketitle

\section{Introduction}

\TBC {\color{red} This is the part that we will write last.}

\medskip

\subsection*{Organization of the paper}

\TBC

\medskip

\begin{acknowledgment}
	\TBC
\end{acknowledgment}

\bigskip

\section{Reformulation of the equation}

Let $\Sigma$ be a smooth, closed surface with a Riemannian metric $g$ and area form $\omega$. Let $G$ be the Green operator of the Laplacian, $\Delta$. For each function, $\varrho \in C^1 \left( (0, T); L^1 \left( \Sigma \right) \right)$, let
\begin{equation}
	 \forall t \in (0, T) : \quad \varrho_t \eqdef \varrho|_{\{ t \} \times \Sigma} \in L^1 \left( \Sigma \right).
\end{equation}
Let us fix $\varrho_0 \in L^1 \left( \Sigma, g \right)$. A function, $\varrho \in C^1 \left( (0, T); L^1 \left( \Sigma \right) \right)$, satisfies the \emph{(parabolic-elliptic) Keller--Segel equations} with initial value $\varrho_0$ if for all $t \in (0, T)$, $\varrho_t \in L_{1, \loc}^2$ and $\varrho$ is (weak) solution to the following system:
\begin{subequations}
\begin{align}
	\del_t \varrho								&= - \Delta \varrho + \rd^* \left( \varrho \rd G \left( \varrho \right) \right), \label{eq:KS_eq} \\
	\lim\limits_{t \rightarrow 0^+} \varrho_t	&= \varrho_0. \label{eq:KS_IV}
\end{align}
\end{subequations}
The convergence in \cref{eq:KS_IV} is in $L^1 \left( \Sigma, g \right)$.

The mass of $\varrho_t$ is
\begin{equation}
	M (t) \eqdef \int\limits_\Sigma \varrho_t \dA.
\end{equation}
Then
\begin{equation}
	\dot{M} (t) = \int\limits_\Sigma \left( - \Delta \varrho + \rd^* \left( \varrho \rd G \left( \varrho \right) \right) \right) \dA = \int\limits_\Sigma \rd^* \left( - \rd \varrho + \varrho \rd G \left( \varrho \right) \right) \dA = 0,
\end{equation}
Thus $M$ constant and hence we drop the $t$-dependence from its notation.

\smallskip

For the rest of the paper, let $A_\Sigma \eqdef \Ar \left( \Sigma, g \right)$. The following lemma recasts \cref{eq:KS_eq,eq:KS_IV} in a simpler form.

\begin{lemma}
	Let $\chi_0 \eqdef \varrho_0 - \tfrac{M}{A_\Sigma} \in L^1 \left( \Sigma \right)$ and $\chi \eqdef \varrho - \tfrac{M}{A_\Sigma} \in C^1 \left( (0, T); L^1 \left( \Sigma \right) \right)$. Then \cref{eq:KS_eq,eq:KS_IV} are equivalent to
	\begin{subequations}
	\begin{align}
		\del_t \chi						&= \tfrac{M}{A_\Sigma} \chi - \Delta \chi + \rd^* \left( \chi \rd G \left( \chi \right) \right), \label{eq:new_KS_eq} \\
		\chi|_{\{ 0 \} \times \Sigma}	&= \chi_0. \label{eq:new_KS_IV}
	\end{align}
	\end{subequations}
\end{lemma}

\begin{proof}
	The equivalency of \cref{eq:KS_IV} and \cref{eq:new_KS_IV} is obvious.

	Note that $G$ annihilates constants and since $\chi$ is orthogonal to constants, we have that $\Delta \left( G \left( \chi \right) \right) = \chi$. Since $\chi = \varrho - \tfrac{M}{A_\Sigma}$, we get, using \cref{eq:KS_eq}, that
	\begin{align}
		\del_t \chi	&= \del_t \left( \varrho - \tfrac{M}{A_\Sigma} \right) \\
					&= \del_t \varrho - 0 \\
					&= - \Delta \varrho + \rd^* \left( \varrho \rd G \left( \varrho \right) \right) \\
					&= - \Delta \left( \tfrac{M}{A_\Sigma} + \chi \right) + \rd^* \left( \left( \tfrac{M}{A_\Sigma} + \chi \right) \rd G \left( \tfrac{M}{A_\Sigma} + \chi \right) \right) \\
					&= - \Delta \chi + \rd^* \left( \left( \tfrac{M}{A_\Sigma} + \chi \right) \rd G \left( \chi \right) \right) \\
					&= - \Delta \chi + \tfrac{M}{A_\Sigma} \rd^* \rd G \left( \chi \right) + \rd^* \left( \chi \rd G \left( \chi \right) \right) \\
					&= \tfrac{M}{A_\Sigma} \chi - \Delta \chi + \rd^* \left( \chi \rd G \left( \chi \right) \right),
	\end{align}
	which completes the proof.
\end{proof}

\smallskip

\begin{remark}
	Let $\lambda_1$ be the smallest nonzero eigenvalue of $\Delta$ and note that quantity $M_\Sigma = \lambda_1 A_\Sigma$ only depends on the geometry of $\left( \Sigma, g \right)$. When $M < M_\Sigma$, then the linear term in \cref{eq:new_KS_eq} is strictly negative definite.
\end{remark}

\bigskip

\section{The generalized Fourier transform}

Since $\Sigma$ is compact, for all $p \in (1, \infty]$, we have that $L^p \left( \Sigma \right) \hookrightarrow L^1 \left( \Sigma, g \right)$. Let us assume now that $\varrho \in C^1 \left( (0, T); L^2 \left( \Sigma \right) \right) \hookrightarrow C^1 \left( (0, T); L^1 \left( \Sigma \right) \right)$ is a solution to the Keller--Segel \cref{eq:KS_eq}.

Let now $\left( \Psi_a \in L^2 \left( \Sigma, g \right) \right)_{a \in \N}$ be an orthonormal eigenbasis of $\Delta$ and
\begin{equation}
	\Delta \Psi_a = \lambda_a \Psi_a.
\end{equation}
Let us order this basis so that
\begin{equation}
	0 = \lambda_0 < \lambda_1 \leqslant \lambda_2 \leqslant \ldots \lambda_a \leqslant \lambda_{a + 1} \leqslant \ldots
\end{equation}
In particular, $\Psi_0 = \tfrac{1}{\sqrt{A_\Sigma}}$.

Assume that $\chi$ is a solution to \cref{eq:new_KS_eq} and write
\begin{equation}
	\forall a \in \N : \ \forall t \in (0, T) : \quad R_a (t) \eqdef \left\langle \Psi_a \middle| \chi|_{\{ t \} \times \Sigma} \right\rangle_{L^2 \left( \Sigma, g \right)}.
\end{equation}
Then for each $a \in \N$, we have that $R_a \in C^1 \left( (0, T); \rl \right)$. Note that $R_0 \equiv 0$. Finally let
\begin{equation}
	\forall a, b, c \in \N : \quad \varphi_{a, b, c} \eqdef \int\limits_\Sigma \Psi_a \Psi_b \Psi_c \dA.
\end{equation}
Then $\varphi$ is a completely symmetric 3-tensor, and
\begin{equation}
	\forall  b, c \in \N : \quad \varphi_{0, b, c} = \tfrac{1}{\sqrt{A_\Sigma}} \delta_{b, c}.
\end{equation}

\smallskip

\begin{theorem}
\label{theorem:ODE}
	Under the above assumptions, the function $\chi$ is a solution to \cref{eq:new_KS_eq} exactly when
	\begin{subequations}
	\begin{align}
		\forall t \in (0, T) &: \quad \left( R_a (t) \right)_{a \in \N_+} \in l^2 \left( \N_+ \right), \label[cond]{cond:R_l2} \\
		\forall a \in \N_+ &: \quad \dot{R}_a = \left( \tfrac{M}{A_\Sigma} - \lambda_a \right) R_a + \sum\limits_{b, c \in \N_+} \frac{\lambda_a - \lambda_b + \lambda_c}{2 \lambda_c} \varphi_{a, b, c} R_b R_c. \label{eq:R_a_eq}
	\end{align}
	\end{subequations}
\end{theorem}

\begin{proof}
	Since $\left( \Psi_a \right)_{a \in \N}$ is an orthonormal basis of $L^2 \left( \Sigma, g \right)$ and for all $t \in (0, T)$, $\varrho_t$ is in $L^2 \left( \Sigma, g \right)$, we get \cref{cond:R_l2}.

	For all $a \in \N_+$, we have $G \left( \Psi_a \right) = \lambda_a^{- 1} \Psi_a$. Using this, the self-adjointness of $\Delta$, and \cref{eq:new_KS_eq}, we get
	\begin{align}
		\dot{R}_a	&= \left\langle \Psi_a \middle| \partial_t \chi \right\rangle_{L^2 \left( \Sigma, g \right)} \\
					&= \left\langle \Psi_a \middle| \tfrac{M}{A_\Sigma} \chi - \Delta \chi + \rd^* \left( \chi \rd G \left( \chi \right) \right) \right\rangle_{L^2 \left( \Sigma, g \right)} \\
					&= \left\langle \tfrac{M}{A_\Sigma} \Psi_a - \Delta \Psi_a \middle| \chi \right\rangle_{L^2 \left( \Sigma, g \right)} + \left\langle \rd \Psi_a \middle| \chi \rd G \left( \chi \right) \right\rangle_{L^2 \left( \Sigma, g \right)} \\
					&= \left( \tfrac{M}{A_\Sigma} - \lambda_a \right) \left\langle \Psi_a \middle| \chi \right\rangle_{L^2 \left( \Sigma, g \right)} + \sum\limits_{b, c \in \N_+} \left\langle \rd \Psi_a \middle| \Psi_b \rd G \left( \Psi_c \right) \right\rangle_{L^2 \left( \Sigma, g \right)} R_b R_c \\
					&= \left( \tfrac{M}{A_\Sigma} - \lambda_a \right) R_a + \sum\limits_{b, c \in \N_+} \left\langle \rd \Psi_a \middle| \Psi_b \rd \Psi_c \right\rangle_{L^2 \left( \Sigma, g \right)} \lambda_c^{- 1} R_b R_c. \label{eq:coeffs}
	\end{align}
	Note that
	\begin{equation}
		\left\langle \rd \Psi_a \middle| \Psi_b \rd \Psi_c \right\rangle_{L^2 \left( \Sigma, g \right)} = \int\limits_\Sigma \Psi_b g \left( \rd \Psi_a, \rd \Psi_c \right) \dA, \label{eq:triple_coeff}
	\end{equation}
	and
	\begin{equation}
		\Delta \left( \Psi_a \Psi_c \right) = \left( \Delta \Psi_a \right) \Psi_c + \Psi_a \left( \Delta \Psi_c \right) - 2 g \left( \rd \Psi_a, \rd \Psi_c \right) = \left( \lambda_a + \lambda_c \right) \Psi_a \Psi_c - 2 g \left( \rd \Psi_a, \rd \Psi_c \right).
	\end{equation}
	Thus
	\begin{equation}
		g \left( \rd \Psi_a, \rd \Psi_c \right) = \frac{\lambda_a + \lambda_c}{2} \Psi_a \Psi_c - \frac{1}{2} \Delta \left( \Psi_a \Psi_c \right).
	\end{equation}
	Plugging the above equation into \cref{eq:triple_coeff} we get
	\begin{align}
		\left\langle \rd \Psi_a \middle| \Psi_b \rd \Psi_c \right\rangle_{L^2 \left( \Sigma, g \right)}	&= \int\limits_\Sigma \Psi_b g \left( \rd \Psi_a, \rd \Psi_c \right) \dA \\
			&= \int\limits_\Sigma \Psi_b \left( \frac{\lambda_a + \lambda_c}{2} \Psi_a \Psi_c - \frac{1}{2} \Delta \left( \Psi_a \Psi_c \right) \right) \dA \\
			&= \frac{\lambda_a + \lambda_c}{2} \varphi_{a, b, c} - \frac{1}{2} \int\limits_\Sigma \left( \Delta \Psi_b \right) \Psi_a \Psi_c \dA \\
			&= \frac{\lambda_a + \lambda_c}{2} \varphi_{a, b, c} - \frac{\lambda_b}{2} \int\limits_\Sigma \Psi_b \Psi_a \Psi_c \dA \\
			&=  \frac{\lambda_a - \lambda_b + \lambda_c}{2} \varphi_{a, b, c}.
	\end{align}
	Inserting this into \cref{eq:coeffs} yields \cref{eq:R_a_eq}.
\end{proof}

\smallskip

\begin{remark}
	The moral of \Cref{theorem:ODE} is that the Keller--Segel equations, which is a (hard) parabolic-elliptic system of partial differential equations, can be transformed (on closed surfaces) into a infinite system of ordinary differential equations, which is potentially easier to handle.

	In the rest of the paper we show that this system can be further simplified under certain extra hypotheses.
\end{remark}

\subsection{Integrating factors}

Using the notation and assumptions of the previous section, let us define (for all $a \in \N_+$ and $n \in \N$)
\begin{equation}
	S_a (t) \eqdef \exp \left( \left( \lambda_a - \tfrac{M}{A_\Sigma} \right) t \right) R_a (t). \label{eq:S_a_def}
\end{equation}
These are the "Fourier" coefficient of $\exp \left( \Delta - \tfrac{M}{A_\Sigma} \id \right) \chi$. Using \cref{eq:R_a_eq}, we get that
\begin{equation}
	\dot{S}_a (t) \eqdef \sum\limits_{b, c \in \N_+} \frac{\lambda_a - \lambda_b + \lambda_c}{2 \lambda_c} \varphi_{a, b, c} \exp \left( \left( \tfrac{M}{A_\Sigma} + \lambda_a - \lambda_b - \lambda_c \right) t \right) S_b (t) R_c (t). \label{eq:S_a_eq}
\end{equation}

\smallskip

\subsection{Analytic solutions}

In order to further simplify \cref{eq:R_a_eq,eq:S_a_eq}, we search for analytic solutions, that is
\begin{equation}
	\forall a \in \N_+ : \forall t \in (0, T): \quad R_a (t) = \sum\limits_{n \in \N} R_{n, a} t^n, \label{eq:analycity}
\end{equation}
and the sums
\begin{equation}
	\forall t \in (0, T): \sum\limits_{n \in \N} \left( R_{n, a} \right)_{a \in \N_+} t^n,
\end{equation}
are assumed to be absolute convergent in $l^2 \left( \N_+ \right)$. Similarly we define
\begin{equation}
	\forall a \in \N_+ : \forall t \in (0, T): \quad S_a (t) = \sum\limits_{n \in \N} S_{n, a} t^n, \label{eq:analycity_for_S}
\end{equation}
The next lemma rewrites \cref{eq:R_a_eq,eq:S_a_eq} in terms of the coefficients $\left( R_{n, a} \right)_{(n, a) \in \N \times \N_+}$ and $\left( S_{n, a} \right)_{(n, a) \in \N \times \N_+}$.

\begin{lemma}
	Under the above assumptions, the function $\chi$ is a $t$-analytic solution to \cref{eq:new_KS_eq} exactly when (for all relevant $a$ and $n$) we have
	\begin{subequations}
	\begin{align}
		R_{n + 1, a}	&= \frac{1}{n + 1} \left( \left( \tfrac{M}{A_\Sigma} - \lambda_a \right) R_{n, a} + \sum\limits_{b, c \in \N_+} \sum\limits_{m = 0}^n \tfrac{\lambda_a - \lambda_b + \lambda_c}{2 \lambda_c} \varphi_{a, b, c} R_{m, b} R_{n - m, c} \right), \label{eq:general_iteration} \\
		\left( R_{n, a} \right)_{a \in \N} &\in l^2 \left( \N \right), \label{eq:regularity} \\
		\limsup\limits_{n \rightarrow \infty} \left( \sum\limits_{a \in \N} R_{n, a}^2 \right)^{\frac{1}{n}} &\leqslant \frac{1}{T}. \label{eq:convergence_radius}
	\end{align}
	\end{subequations}
	If $\exp \left( \Delta - \tfrac{M}{A_\Sigma} \id \right) \chi$ is also defined for all $t \in (0, T)$, then
	\begin{equation}
		S_{n + 1, a} = \frac{1}{n + 1} \sum\limits_{b, c \in \N_+} \sum\limits_{x + y + z = n} \tfrac{\lambda_a - \lambda_b + \lambda_c}{2 \lambda_c} \varphi_{a, b, c} S_{x, b} S_{y, c} \tfrac{\left( \tfrac{M}{A_\Sigma} + \lambda_a - \lambda_b - \lambda_c \right)^z}{z!}. \label{eq:general_iteration_for_S}
	\end{equation}
\end{lemma}

\begin{proof}
	Inserting \cref{eq:analycity} into \cref{eq:R_a_eq} yields \cref{eq:general_iteration}. The \cref{eq:regularity,eq:convergence_radius} are necessary (and, in fact, sufficient) to have that the convergence radius of the Taylor series of $\varrho$ in the $L^2$ topology is at least $T$. The claim about \cref{eq:general_iteration_for_S} is straightforward to check.
\end{proof}

In the following two sections we investigate two special cases when iteration in \cref{eq:general_iteration} exists for all $a$ and $n$.

\bigskip

\section{Round spheres}
\label{sec:sphere}

Let $\Sigma$ be the 2-sphere and $g$ be the round metric of radius $r$. Then we have that $\left( \Psi_a \right)_{a \in \N}$ are the spherical harmonics. Then $A_\Sigma = 4 \pi r^2$. In fact, after relabeling them, we can write the eigenvalue has the form $\lambda_{l, m} \eqdef \tfrac{l (l + 1)}{r^2}$, where $l \in \N$ and $M$ is any integer satisfying $|m| \leqslant l$. Let us now write
\begin{equation}
	\Psi_l^m \eqdef \Psi_{(l, m)}, \quad \& \quad R_{l, a}^m \eqdef R_{(l, m), a}, \quad \& \quad \varphi_{l_1, l_2, l_3}^{m_1, m_2, m_3} \eqdef \varphi_{(l_1, m_1), (l_2, m_2), (l_3, m_3)}.
\end{equation}
Using this new set of indices and notation, we can rewrite \cref{eq:general_iteration} as (for all $l \in \N_+$)
\begin{align}
	R_{l, n + 1}^m	&= \frac{1}{n + 1} \left( \tfrac{M}{4 \pi r^2} - \tfrac{l (l + 1)}{r^2} \right) R_{(l, m), n} \\
					& \quad + \sum\limits_{l_1, l_2 \in \N_+} \sum\limits_{m_1 = - l_1}^{l_1} \sum\limits_{m_2 = - l_2}^{l_2} \sum\limits_{x = 0}^n \tfrac{l (l + 1) - l_1 (l_1 + 1) + l_2 (l_2 + 1)}{2 (n + 1) l_2 (l_2 + 1)} \varphi_{l, l_1, l_2}^{m, m_1, m_2} R_{l_1, x}^{m_1} R_{l_2, n - x}^{m_2}. \label{eq:iteration_on_S2_naive}
\end{align}
Using the Clebsch--Gordan Theorem, we have that if $l_1 \geqslant l_2 + l_3$ or $l_1 \leqslant |l_2 - l_3|$, then for all $m_1, m_2$, and $m_3$, we have $\varphi_{l_1, l_2, l_3}^{m_1, m_2, m_3} = 0$. Thus \cref{eq:iteration_on_S2_naive} becomes
\begin{align}
	R_{l, n + 1}^m	&= \frac{1}{n + 1} \left( \tfrac{M}{4 \pi r^2} - \tfrac{l (l + 1)}{r^2} \right) R_{l, n}^m \\
					& \quad + \sum\limits_{l_1 \in \N_+} \sum\limits_{l_2 = \max \left( \{ 1, |l - l_1| \} \right)}^{l + l_1} \sum\limits_{m_1 = - l_1}^{l_1} \sum\limits_{m_2 = - l_2}^{l_2} \sum\limits_{x = 0}^n \tfrac{l (l + 1) - l_1 (l_1 + 1) + l_2 (l_2 + 1)}{2 (n + 1) l_2 (l_2 + 1)} \varphi_{l, l_1, l_2}^{m, m_1, m_2} R_{l_1, x}^{m_1} R_{l_2, n - x}^{m_2}. \label{eq:iteration_on_S2}
\end{align}

Before prove the main result of this section, let us make the following definition:
\begin{equation}
	\forall n \in \N : \quad Z_n \eqdef \left\{ \ (l, m) \in \N \times \Z \ \middle| \ R_{l, n}^m \neq 0 \ \right\}.
\end{equation}

\begin{theorem}
	Assume that $Z_0$ is finite. Then for all $(l, m) \in \N \times \Z$ and $n \in \N$, $R_{l, n}^m$ exists. Furthermore, $Z_n$ is also finite, and \cref{eq:iteration_on_S2} becomes
	\begin{align}
		R_{l, n + 1}^m	&= \frac{1}{n + 1} \left( \tfrac{M}{4 \pi r^2} - \tfrac{l (l + 1)}{r^2} \right) R_{l, n}^m \\
						& \quad + \sum\limits_{x = 0}^n \sum\limits_{\substack{(l_1, m_1) \in Z_x \\ (l_2, m_2) \in Z_{n - x}}} \tfrac{l (l + 1) - l_1 (l_1 + 1) + l_2 (l_2 + 1)}{2 (n + 1) l_2 (l_2 + 1)} \varphi_{l, l_1, l_2}^{m, m_1, m_2} R_{l_1, x}^{m_1} R_{l_2, n - x}^{m_2}. \label{eq:iteration_on_S2_finite}
	\end{align}
\end{theorem}

\begin{proof}
	Let us prove by induction.

	Since the claim for $n = 0$ is the hypothesis of the theorem, we only need to assume that we have already proven the claim for all nonnegative integers up to and including $n \in \N_+$.

	The right-hand side of \cref{eq:iteration_on_S2} contains coefficients of the form $R_{l_1, x}^{m_1}$ and $R_{l_2, n - x}^{m_2}$, with $0 \leqslant x \leqslant n$, we have that $R_{l_1, x}^{m_1}$ unless $(l_1, m_1) \in Z_x$ and $R_{l_2, n - x}^{m_2}$ unless $(l_2, m_2) \in Z_{n - x}$. Thus for every $x$ we have the contribution of a finite sum, and we only consider finitely many $x$'s, this proves \cref{eq:iteration_on_S2_finite}.

	Since now $R_{l, n + 1}^m$ is expressed as a finite sum, it exists, which concludes the proof.
\end{proof}

\smallskip

\begin{remark}
	Similar results can be proven about the corresponding $S_{n, a}$ coefficients, which we omit here.
\end{remark}

\bigskip

\section{Flat tori}
\label{sec:tori}

Let now $\Sigma$ be a flat torus. Thus, without any loss of generality, we can assume that there are vectors
\begin{equation}
	\underline{e}_1 = \begin{pmatrix} L_1 \\ 0 \end{pmatrix}, \quad \& \quad \underline{e}_2 = \begin{pmatrix} L_2 \cos (\theta) \\ L_2 \sin (\theta) \end{pmatrix}.
\end{equation}
such that, if we define the \emph{lattice} $\Lambda \eqdef \Z \underline{e}_1 \oplus \Z \underline{e}_2$, then
\begin{equation}
	\Sigma = \rl^2 / \Lambda.
\end{equation}
Note that $A_\Sigma = L_1 L_2 \sin (\theta)$. Let the \emph{dual} lattice be
\begin{equation}
	\Lambda^* \eqdef \left\{ \ \underline{k} \in \rl^2 \ \middle| \ \forall \underline{x} \in \Lambda : \ \underline{k} \cdot \underline{x} \in \Z \ \right\}.
\end{equation}
It is easy to see that if
\begin{equation}
	\underline{f}_1 \eqdef \begin{pmatrix} \tfrac{1}{L_1} \\ - \tfrac{\cot (\theta)}{L_1} \end{pmatrix}, \quad \& \quad \underline{f}_2 = \begin{pmatrix} 0 \\ \tfrac{1}{L_2 \sin (\theta)} \end{pmatrix}.
\end{equation}
then $\underline{e}_i \cdot \underline{f}_j = \delta_{i, j}$ and thus
\begin{equation}
	\Lambda^* = \Z \underline{f}_1 \oplus \Z \underline{f}_2.
\end{equation}

Now let
\begin{equation}
	\forall \underline{x} \in \Sigma : \forall \underline{k} \in \Lambda^* : \quad \Psi_{\underline{k}} \left( \underline{x} \right) \eqdef \frac{1}{\sqrt{A_\Sigma}} e^{2 \pi i \underline{k} \cdot \underline{x}}.
\end{equation}
Then $\left( \Psi_{\underline{k}} \right)_{\underline{k} \in \Lambda^*}$ is an orthonormal basis for the \emph{complex} Hilbert space $L_\cx^2 \left( \Sigma, g \right)$. Furthermore, note that $\Psi_{\underline{k}} = \overline{\Psi_{- \underline{k}}}$. Finally, note that
\begin{equation}
	\Delta \Psi_{\underline{k}} = 4 \pi^2 \left| \underline{k} \right|^2 \Psi_{\underline{k}},
\end{equation}
thus $\left( \Psi_{\underline{k}} \right)_{\underline{k} \in \Lambda^*}$ is an eigenbasis for the Laplacian, albeit a complex one. The corresponding spectrum is $\left( 4 \pi^2 \left| \underline{k} \right|^2 \right)_{\underline{k} \in \Lambda^*}$.

Function on $\Sigma$ can be viewed as $\Lambda$-periodic functions on $\rl^2$, thus if $\chi$ is an ($L_\cx^2$) function on $\Sigma$, then we use Fourier decomposition to get:
\begin{equation}
	\forall \underline{k} \in \Lambda^* : \quad R_{\underline{k}} \eqdef \frac{1}{\sqrt{A_\Sigma}} \int\limits_\Sigma e^{- 2 \pi i \underline{k} \cdot \underline{x}} \chi \left( \underline{x} \right) \dA \left( \underline{x} \right), \quad \Leftrightarrow \quad \chi = \sum\limits_{\underline{k} \in \Lambda^*} R_{\underline{k}} \Psi_{\underline{k}}.
\end{equation}
Note again that $R_{\underline{0}} \equiv 0$. If $\chi$ is real, then $R_{\underline{k}} = \overline{R_{- \underline{k}}}$. In this section we slightly deviate from our previous method and use the above complex basis and coefficients.

The ideas and proofs of the previous sections still apply, and if $\chi \in C^1 \left( (0, T), L_\cx^2 \left( \Sigma, g \right) \right)$, then we can define the coefficients functions $R_{\underline{k}}, S_{\underline{k}} \in C_\cx^1 \left( \Sigma \right)$, so that
\begin{align}
	\chi \left( t, \underline{x} \right)																	&= \sum\limits_{n \in \N} \sum\limits_{\underline{k} \in \Lambda^* - \left\{ \underline{0} \right\}} R_{\underline{k}} (t) \Psi_{\underline{k}} \left( \underline{x} \right), \\
	\exp \left( \left( \Delta - \tfrac{M}{A_\Sigma} \right) t \right) \chi \left( t, \underline{x} \right)	&= \sum\limits_{n \in \N} \sum\limits_{\underline{k} \in \Lambda^* - \left\{ \underline{0} \right\}} S_{\underline{k}} (t) \Psi_{\underline{k}} \left( \underline{x} \right).
\end{align}
If, furthermore, $\chi$ is a solution to the Keller--Segel \cref{eq:new_KS_eq}, then we get (after a straightforward computation) that for all $\underline{k} \in \Lambda^* - \left\{ \underline{0} \right\}$
\begin{align}
	\dot{R}_{\underline{k}}	&= \left( \tfrac{M}{A_\Sigma} - 4 \pi^2 \left| \underline{k} \right|^2 \right) R_{\underline{k}} + \sum\limits_{\underline{l} \in \Lambda^* - \left\{ \underline{0}, \underline{k} \right\}} \frac{\underline{k} \cdot \underline{l}}{\left| \underline{l} \right|^2 \sqrt{A_\Sigma}} R_{\underline{l}} R_{\underline{k} - \underline{l}}, \\
	\dot{S}_{\underline{k}}	&= \sum\limits_{\underline{l} \in \Lambda^* - \left\{ \underline{0}, \underline{k} \right\}} \frac{\underline{k} \cdot \underline{l}}{\left| \underline{l} \right|^2 \sqrt{A_\Sigma}} \exp \left( \tfrac{M}{A_\Sigma} + 8 \pi^2 \left( \underline{k} - \underline{l} \right) \cdot \underline{l} \right) S_{\underline{l}} S_{\underline{k} - \underline{l}}
\end{align}
Finally, if $\chi$ is analytic in $t$, and we define $R_{n, \underline{k}}, S_{n, \underline{k}} \in \cx$ through
\begin{align}
	R_{\underline{k}} (t)	&= \sum\limits_{n \in \N} R_{n, \underline{k}} t^n, \\
	S_{\underline{k}} (t)	&= \sum\limits_{n \in \N} S_{n, \underline{k}} t^n,
\end{align}
then we get the corresponding iteration is (for all $n \in \N$ and $\underline{k} \in \Lambda^* - \left\{ \underline{0} \right\}$)
\begin{subequations}
\begin{align}
	R_{n + 1, \underline{k}} &= \frac{1}{n + 1} \left( \left( \tfrac{M}{A_\Sigma} - 4 \pi^2 \left| \underline{k} \right|^2 \right) R_{n, \underline{k}} + \sum\limits_{\underline{l} \in \Lambda^* - \left\{ \underline{0}, \underline{k} \right\}} \sum\limits_{m = 0}^n \frac{\underline{k} \cdot \underline{l}}{\left| \underline{l} \right|^2 \sqrt{A_\Sigma}} R_{m, \underline{l}} R_{n - m, \underline{k} - \underline{l}} \right), \label{eq:iteration_on_torus} \\
	S_{n + 1, \underline{k}} &= \frac{1}{n + 1} \sum\limits_{\underline{l} \in \Lambda^* - \left\{ \underline{0}, \underline{k} \right\}} \sum\limits_{a + b + c = n} \frac{\underline{k} \cdot \underline{l}}{\left| \underline{l} \right|^2 \sqrt{A_\Sigma}} S_{a, \underline{l}} S_{b, \underline{k} - \underline{l}} \frac{\left( \tfrac{M}{A_\Sigma} + 8 \pi^2 \left( \underline{k} - \underline{l} \right) \cdot \underline{l} \right)^c}{c!}, \label{eq:iteration_on_torus_for_S}
\end{align}
\end{subequations}

As in \Cref{sec:sphere}, let us make the following definition: For all $n \in \N$, let
\begin{align}
	Z_n			&\eqdef \left\{ \ \underline{k} \in \Lambda^* - \left\{ \underline{0} \right\} \ \middle| \ R_{n, \underline{k}} \neq 0 \ \right\} \cup \left\{ \underline{0} \right\}, \\
	\hat{Z}_n	&\eqdef \left\{ \ \underline{k} \in \Lambda^* - \left\{ \underline{0} \right\} \ \middle| \ S_{n, \underline{k}} \neq 0 \ \right\} \cup \left\{ \underline{0} \right\}.
\end{align}
Furthermore, let
\begin{align}
	d_n			&\eqdef \max \left( \left\{ \ \left| \underline{k} \right| \ \middle| \ \underline{k} \in Z_n \ \right\} \right), \\
	\hat{d}_n	&\eqdef \max \left( \left\{ \ \left| \underline{k} \right| \ \middle| \ \underline{k} \in \hat{Z}_n \ \right\} \right).
\end{align}
Note that $Z_0 = \hat{Z}_0$, and thus $d_0 = \hat{d}_0$.

\begin{theorem}
	\label{theorem:torus_finite}
	Assume that $Z_0$ is finite. Then for all $\underline{k} \in \Lambda^* - \left\{ \underline{0} \right\}$ and $n \in \N$, $R_{n, \underline{k}}$ exists. Moreover, $Z_n$ is also finite and \cref{eq:iteration_on_torus,eq:iteration_on_torus_for_S} become, for all $\underline{k} \in \Lambda^* - \left\{ \underline{0} \right\}$
	\begin{align}
		 R_{n + 1, \underline{k}}	&= \frac{1}{n + 1} \left( \left( \tfrac{M}{A_\Sigma} - 4 \pi^2 \left| \underline{k} \right|^2 \right) R_{n, \underline{k}} + \sum\limits_{m = 0}^n \sum\limits_{\substack{\underline{l} \in Z_m \\ \underline{k} - \underline{l} \in Z_{n - m}}} \frac{\underline{k} \cdot \underline{l}}{\left| \underline{l} \right|^2 \sqrt{A_\Sigma}} R_{m, \underline{l}} R_{n - m, \underline{k} - \underline{l}} \right), \label{eq:iteration_on_torus_finite} \\
		 S_{n + 1, \underline{k}}	&= \frac{1}{n + 1} \sum\limits_{x + y + z = n} \sum\limits_{\substack{\underline{l} \in \hat{Z}_a \\ \underline{k} - \underline{l} \in \hat{Z}_b}} \tfrac{\underline{k} \cdot \underline{l}}{\left| \underline{l} \right|^2 \sqrt{A_\Sigma}} S_{a, \underline{l}} S_{b, \underline{k} - \underline{l}} \tfrac{\left( \tfrac{M}{A_\Sigma} + 8 \pi^2 \left( \underline{k} - \underline{l} \right) \cdot \underline{l} \right)^z}{z!}. \label{eq:iteration_on_torus_finite_for_S}
	\end{align}
	Moreover
	\begin{align}
		d_n, \hat{d}_n \leqslant (n + 1) d_0, \label[ineq]{ineq:d_n}
	\end{align}
\end{theorem}

\begin{proof}
	We only prove the claim for the $R_{n, \underline{k}}$ coefficients. The proof is analogous for the $S_{n, \underline{k}}$ coefficients.

	Let us prove by induction. The claim for $n = 0$ is the hypothesis of the theorem.

	Let us now assume that all nonnegative integers, $m$, up to but not including $n \in \N_+$, we have that for all $\underline{k} \in \Lambda^* - \left\{ \underline{0} \right\}$, $R_{m, \underline{k}}$ exists, $Z_m$ is finite, \cref{eq:iteration_on_torus_finite} holds (with $n$ replaced by $m$), and $d_m \leqslant (m + 1) d_0$.

	Since \cref{eq:iteration_on_torus_finite} holds when $n$ is replaced by $m = n - 1$, we have that
	\begin{equation}
		\forall \underline{k} \in \Lambda^* - \left\{ \underline{0} \right\} : R_{n, \underline{k}} = \frac{1}{n} \left( \left( \tfrac{M}{A_\Sigma} - 4 \pi^2 \left| \underline{k} \right|^2 \right) R_{n - 1, \underline{k}} + \sum\limits_{m = 0}^{n - 1} \sum\limits_{\substack{\underline{l} \in Z_m \\ \underline{k} - \underline{l} \in Z_{n - 1 - m}}} \frac{\underline{k} \cdot \underline{l}}{\left| \underline{l} \right|^2} R_{m, \underline{l}} R_{n - 1 - m, \underline{k} - \underline{l}} \right). \label{eq:iteration_for_n}
	\end{equation}
	Let
	\begin{equation}
		Z_n^\prime \eqdef \bigcup\limits_{m = 0}^{n - 1} \left( Z_m + Z_{n - 1 - m} \right),
	\end{equation}
	By assumption, $Z_n^\prime$ is a finite union of finite sets, so it is also finite. Pick any $\underline{k} \in Z_n^\prime$. Since $Z_{n - 1} \subseteq Z_n^\prime$, we have that the first term in \cref{eq:iteration_for_n} vanishes. By construction, there cannot exist $m \in [0, n - 1] \cap \N$ and $\underline{l} \in Z_m$, such that $\underline{k} - \underline{l} \in Z_{n - 1 - m}$, so all summands of the second term of \cref{eq:iteration_for_n} also vanish, hence $Z_n \subseteq Z_n^\prime$, thus $Z_n$ is finite.

	Using the finiteness of $Z_m$, now for all $m \in [0, n] \cap \N$, and \cref{eq:iteration_on_torus} implies \cref{eq:iteration_on_torus_finite}.

	Since now $R_{n, \underline{k}}$ is expressed as a finite sum, it exists.

	Finally, if $\underline{k} \in Z_n - Z_{n - 1}$, then there exist $m \in [0, n - 1] \cap \N$, $\underline{l}_1 \in Z_m$, and $\underline{l}_2 \in Z_{n - m}$, such that $\underline{k} = \underline{l}_1 + \underline{l}_2$. Thus
	\begin{equation}
		\left| \underline{k} \right|  = \left| \underline{l}_1 + \underline{l}_2 \right| \leqslant \left| \underline{l}_1 \right| + \left| \underline{l}_2 \right| \leqslant d_m + d_{n - m} \leqslant (m + 1) d_0 + (n - m + 1) d_0 = \left( (n + 1) + 1 \right) d_0,
	\end{equation}
	which concludes the proof.
\end{proof}

\smallskip

\begin{corollary}
	Under the assumptions of \Cref{theorem:torus_finite}, let $d \eqdef d_0$. Then we have
	\begin{equation}
		|Z_n| \leqslant 2 \pi n^2 d^2 A_\Sigma.
	\end{equation}
\end{corollary}

\smallskip

We are now ready to prove our main theorem.

\begin{theorem}
	Let $\varrho_0 \in L^1 \left( \Sigma, g \right)$ be such that it has only finitely many nonzero Fourier coefficients, that is, $Z_0$ is finite. Then there exists $T > 0$, and $\varrho \in C^\omega \left( (0 , T), L^2 \left( \Sigma, g \right) \right)$, such that it solves \cref{eq:KS_eq,eq:KS_IV}.
\end{theorem}

\begin{proof}
	By \Cref{theorem:torus_finite}, for all $n \in \N$ and $\underline{k} \in \Lambda^* - \left\{ \underline{0} \right\}$ the $R_{n, \underline{k}}$ coefficients exist. We first prove that there exists $C$, such that for all $n \in \N_+$ and $\underline{k} \in \Lambda^* - \left\{ \underline{0} \right\}$
	\begin{equation}
		\left| R_{n, \underline{k}} \right| \leqslant C^{n + 1} \tfrac{n^n}{(n + 1)!} \exp \left( \tfrac{\left| \underline{k} \right|}{C} \right), \label[ineq]{ineq:R_bound}
	\end{equation}
	and thus
	\begin{equation}
		\limsup\limits_{n \in \N} \sqrt[n]{\left\| R_n \right\|_{l^1 \left( \Lambda^* \right)}} = \lim\limits_{n \rightarrow \infty} \sqrt[n]{\left\| R_n \right\|_{l^1 \left( \Lambda^* \right)}} \leqslant \frac{C}{e} \defeq \frac{1}{T}. \label[ineq]{ineq:T_def}
	\end{equation}
	One can pick $C > 0$, so that \cref{ineq:R_bound} holds for $n = 0$. Let us assume that we have already proven \cref{eq:R_bound} for all nonnegative integers up to and including $n \in \N_+$. Then, by \Cref{theorem:torus_finite}, for $n + 1$, we only need to consider $\underline{k} \in Z_{n + 1}$, in particular, $\left| \underline{k} \right| \leqslant (n + 2) d_0$
	\begin{align}
		\left| \tfrac{\tfrac{M}{A_\Sigma} - 4 \pi^2 \left| \underline{k} \right|^2}{n + 1} R_{n, \underline{k}} \right|	&\leqslant \tfrac{C \left| \underline{k} \right|^2}{2} C^{n + 1} \tfrac{n^n}{(n + 1)!} \exp \left( \tfrac{\left| \underline{k} \right|}{C} \right)
	\end{align}


	, using \cref{eq:iteration_on_torus_finite}, we have

	---

	*This part is \TBC *

	---

	Now for all $t \in (0, T)$ and $\underline{x} \in \Sigma$, let
	\begin{equation}
		\varrho \left( t, \underline{x} \right) \eqdef \tfrac{M}{A_\Sigma} + \sum\limits_{\underline{k} \in \Lambda^* - \left\{ \underline{0} \right\}} S_{\underline{k}} (t) \exp \left( \left( \tfrac{M}{A_\Sigma} - 4 \pi^2 \left| \underline{k} \right|^2 \right) t \right) \tfrac{e^{2 \pi i \underline{k} \cdot \underline{x}}}{\sqrt{A_\Sigma}}. \label{eq:varrho_def}
	\end{equation}
	Now note that if $\underline{k} \in \hat{Z}_n - \hat{Z}_{n - 1}$, then
	\begin{equation}
		\left| S_{\underline{k}} (t) - S_{\underline{k}} (0) \right| \leqslant C \frac{t^n}{T - |t|}.
	\end{equation}
	In particular, if $n > 0$, then
	\begin{equation}
		\left| S_{\underline{k}} (t) \right| \leqslant C \frac{t^n}{T - |t|}.
	\end{equation}
	Thus
	\begin{align}
		\sum\limits_{\underline{k} \in \Lambda^* - \left\{ \underline{0} \right\}} \left| S_{\underline{k}} (t) \exp \left( \left( \tfrac{M}{A_\Sigma} - 4 \pi^2 \left| \underline{k} \right|^2 \right) t \right) \tfrac{e^{2 \pi i \underline{k} \cdot \underline{x}}}{\sqrt{A_\Sigma}} \right|	&\leqslant \sum\limits_{\underline{k} \in Z_0} \left| S_{\underline{k}} (t) \right| \exp \left( \left( \tfrac{M}{A_\Sigma} - 4 \pi^2 \left| \underline{k} \right|^2 \right) t \right) \tfrac{1}{\sqrt{A_\Sigma}} \\
		& \quad + \sum\limits_{n \in \N_+} \sum\limits_{\underline{k} \in \hat{Z}_n - \hat{Z}_{n - 1}} \left| S_{\underline{k}} (t) \right| \exp \left( \left( \tfrac{M}{A_\Sigma} - 4 \pi^2 \left| \underline{k} \right|^2 \right) t \right) \tfrac{1}{\sqrt{A_\Sigma}} \\
		&\leqslant \frac{C}{(T - |t|) \sqrt{A_\Sigma}} \exp \left( \tfrac{M}{A_\Sigma} t \right) \left( \| \varrho_0 \|_{L^1 \left( \Sigma, g \right)} + \sum\limits_{n \in \N} |\hat{Z}_n| t^n \right) \\
		&\leqslant \frac{C}{(T - |t|) \sqrt{A_\Sigma}} \exp \left( \tfrac{M}{A_\Sigma} T \right) \left( \| \varrho_0 \|_{L^1 \left( \Sigma, g \right)} + C \frac{T^2}{(T - |t|)^3} \right).
	\end{align}

	Thus right-hand side of \cref{eq:varrho_def} is absolute convergent for all $t \in (- T, T)$. In particular
	\begin{align}
		\lim\limits_{t \rightarrow 0^+} \varrho \left( t, \underline{x} \right)	&= \varrho \left( 0, \underline{x} \right) \\
		&= \tfrac{M}{A_\Sigma} + \sum\limits_{\underline{k} \in \Lambda^* - \left\{ \underline{0} \right\}} S_{\underline{k}} (0) \exp \left( \left( \tfrac{M}{A_\Sigma} - 4 \pi^2 \left| \underline{k} \right|^2 \right) 0 \right) \tfrac{e^{2 \pi i \underline{k} \cdot \underline{x}}}{\sqrt{A_\Sigma}} \\
		&= \tfrac{M}{A_\Sigma} + \sum\limits_{\underline{k} \in \Lambda^* - \left\{ \underline{0} \right\}} R_{\underline{k}} (0) \tfrac{e^{2 \pi i \underline{k} \cdot \underline{x}}}{\sqrt{A_\Sigma}} \\
		&= \varrho_0 \left( \underline{x} \right),
	\end{align}
	which concludes the proof.
\end{proof}

\section{Exact solutions}

	%=========================
	%\bibliography{references}
	%=========================

\end{document}